\documentclass[a4papper,12pt]{article}
\usepackage[brazil]{babel}
\usepackage[utf8x]{inputenc}
\usepackage[T1]{fontenc}
\usepackage[a4paper]{geometry}
\geometry{verbose,tmargin=2cm,bmargin=2cm,lmargin=2cm,rmargin=2cm}
\usepackage{graphicx}
\usepackage{wrapfig}
\usepackage{setspace}
\usepackage{amsmath,amssymb,amsfonts} 
\onehalfspacing

\begin{document}


Encontre o comprimento arco da curva dada por $\mathbf{r}(t) = t \,\mathbf{i} + \dfrac{4}{3}t^{3/2}\,\mathbf{j} + \dfrac{1}{2}t^{2}  \,\mathbf{k}$ entre $t=0$ e $t=2$.


Da definição da função $\mathbf{r}(t)$ identificamos as funções componentes e calculamos suas derivas:
          \begin{eqnarray*}
            x'(t) &=& 1\\
            y'(t) &=& 2t^{1/2}\\
            z'(t) &=& t
          \end{eqnarray*}
        
        dessa forma
        
          \begin{eqnarray*}
            L &=& \int_0^2 \sqrt{\left( 1 \right)^2 + \left( 2t^{1/2} \right)^2 + \left( t\right)^2}\,\,\,\,dt\\
            &=& \int_0^2 \sqrt{1 + 4t + t^2}\,\,dt\\
            &=& \int_0^2 \sqrt{(t+2)^2 -3}\,\,dt\\
          \end{eqnarray*}
a solução dessa equação requer utilizar as seguentes substituições:
          \begin{eqnarray*}
            n  =& t+2\\
            dn =& dt 
          \end{eqnarray*}
        vamos solucionar a integral indefinida, assim a integral muda para
          \begin{eqnarray*}
            I =& \int \sqrt{n^2 - 3} dn\\
          \end{eqnarray*}
        utilizando substituição trigonométrica, vemos que colocando $u$ na hipotenusa e $\sqrt{3}$ no cateto adjacente, então
          \begin{eqnarray*}
          \sqrt{n^2 - 3} =& \sqrt{3}\tan \theta\\
          \dfrac{n}{\sqrt{3}} =&\sec \theta\\
          dn =& \sqrt{3} \sec \theta \tan \theta d\theta
          \end{eqnarray*}
        de onde
        \[
          \begin{array}{rl}
          I =& 3\int \sec \theta \tan ^2 \theta d\theta \nonumber\\
            =&  3\int \sec \theta \left( \sec ^2 \theta - 1\right)d\theta \nonumber\\
            =&  3\int \sec ^3 \theta d\theta - 3\int \sec \theta d\theta \nonumber
          \end{array}
        \]
        resolvendo cada uma das integrais:
        \[
          \begin{array}{rl}
            I_2 = \int \sec \theta d\theta 
             =&\int \sec \theta \dfrac{\sec \theta + \tan \theta}{\sec \theta + \tan \theta} d\theta \nonumber\\
            =& \dfrac{\sec ^2 \theta + \sec \theta \tan \theta}{\sec \theta + \tan \theta} d\theta \nonumber
          \end{array}
        \]
        fazemos
        \[
          \begin{array}{rl}
            u  =& \sec \theta + \tan \theta \nonumber\\
            du =& \left( \sec \theta  \tan \theta + \sec^2 \theta\right) d\theta \nonumber
          \end{array}
        \]
        assim
        \[
          \begin{array}{r l}
            I_2^* =& \int \dfrac{1}{u}\,du\nonumber\\
            I_2=& \ln \left| \sec \theta + \tan \theta \right| + C_1\nonumber
          \end{array}
        \]
        A I integral:
        \[
          \begin{array}{rl}
          I_1 =& \sec ^3 \theta d\theta\nonumber\\
          \end{array}
        \]
        integrando por partes:
        \[
          \begin{array}{rlcrl}
          u =& \sec \theta &\;\;\;\;\ & dv =& \sec ^2 \theta d\theta\nonumber\\
          du =& \sec \theta  \tan \theta d\theta & & v =& \tan \theta \nonumber
          \end{array}
        \]
        assim
      \[
          \begin{array}{rl}
            \sec ^3 \theta d\theta  =& \sec \theta  \tan \theta - \int \left( \sec ^3 \theta - \sec \theta \right)d\theta \nonumber\\
            2\sec ^3 \theta d\theta  =& \sec \theta  \tan \theta - \int \sec \theta d\theta \nonumber\\
            \sec ^3 \theta d\theta  =& \dfrac{1}{2}\sec \theta  \tan \theta - \dfrac{1}{2} \ln \left| \sec \theta + \tan \theta \right| + C_2\nonumber
          \nonumber
          \end{array}
        \]
        assim a integral original fica
        \[
          I = \dfrac{1}{2}\sec \theta  \tan \theta + \dfrac{1}{2} \ln \left| \sec \theta + \tan \theta \right| + C\nonumber
        \]
        como
        \[
          \sec \theta = \dfrac{t+2}{\sqrt{3}}\;\;\;\;\;\; \tan \theta = \dfrac{\sqrt{(t+2)^2 -3}}{\sqrt{3}}\nonumber
        \]
        então
        \[
          \int \sqrt{(t+2)^2 -3}\,\,dt = \dfrac{\left(t+2 \right)\sqrt{(t+2)^2 -3}}{2} - \frac{3}{2}\left| \dfrac{t+2+\sqrt{(t+2)^2 -3}}{\sqrt{3}} \;\;\;\right|\nonumber
        \]
\end{document}
