%% LyX 2.0.6 created this file.  For more info, see http://www.lyx.org/.
%% Do not edit unless you really know what you are doing.
\documentclass[12pt,brazil,a4papper]{article}
\usepackage[T1]{fontenc}
\usepackage[utf8x]{inputenc}
\usepackage{geometry}
\geometry{verbose,tmargin=2cm,bmargin=2cm,lmargin=2cm,rmargin=2cm}
\usepackage{amsmath}
\usepackage{amssymb}
\usepackage{setspace}
\usepackage{esint}
\onehalfspacing

\makeatletter
%%%%%%%%%%%%%%%%%%%%%%%%%%%%%% User specified LaTeX commands.

\usepackage[brazil]{babel}


\usepackage{amsfonts}

\makeatother

\usepackage{babel}
\begin{document}
\begin{align*}
\vec{\nabla}\times\mathbf{F}= & \begin{bmatrix}\mathbf{i} & \mathbf{j} & \mathbf{k}\\
\dfrac{\partial\;}{\partial x} & \dfrac{\partial\;}{\partial y} & \dfrac{\partial\;}{\partial z}\\
\dfrac{-x}{\sqrt{x^{2}+y^{2}}} & \dfrac{-y}{\sqrt{x^{2}+y^{2}}} & 0
\end{bmatrix}\\
\\
= & \left[\dfrac{\partial\;}{\partial y}0-\dfrac{\partial\;}{\partial z}\left(\dfrac{-y}{\sqrt{x^{2}+y^{2}}}\right)\right]\;\mathbf{i}-\left[\dfrac{\partial\;}{\partial x}0-\dfrac{\partial\;}{\partial z}\left(\dfrac{-x}{\sqrt{x^{2}+y^{2}}}\right)\right]\;\mathbf{j}+\\
 & \left[\dfrac{\partial\;}{\partial x}\left(\dfrac{-y}{\sqrt{x^{2}+y^{2}}}\right)-\dfrac{\partial\;}{\partial y}\left(\dfrac{-x}{\sqrt{x^{2}+y^{2}}}\right)\right]\;\mathbf{k}\\
\\
= & 0\,\mathbf{i}-0\,\mathbf{j}+\left(\dfrac{-xy\left(x^{2}+y^{2}\right)^{-\frac{1}{2}}}{x^{2}+y^{2}}-\dfrac{-xy\left(x^{2}+y^{2}\right)^{-\frac{1}{2}}}{x^{2}+y^{2}}\right)\,\mathbf{k}\\
= & 0\,\mathbf{i}-0\,\mathbf{j}+0\,\mathbf{k}
\end{align*}


Calcule o campo rotacional de $\mathbf{F}=\dfrac{-y\,\mathbf{i}+x\,\mathbf{j}}{\sqrt{x^{2}+y^{2}}}$

\begin{align*}
\vec{\nabla}\times\mathbf{F}= & \begin{bmatrix}\mathbf{i} & \mathbf{j} & \mathbf{k}\\
\dfrac{\partial\;}{\partial x} & \dfrac{\partial\;}{\partial y} & \dfrac{\partial\;}{\partial z}\\
\dfrac{-y}{\sqrt{x^{2}+y^{2}}} & \dfrac{x}{\sqrt{x^{2}+y^{2}}} & 0
\end{bmatrix}\\
\\
= & \left[\dfrac{\partial\;}{\partial y}0-\dfrac{\partial\;}{\partial z}\left(\dfrac{x}{\sqrt{x^{2}+y^{2}}}\right)\right]\;\mathbf{i}-\left[\dfrac{\partial\;}{\partial x}0-\dfrac{\partial\;}{\partial z}\left(\dfrac{-y}{\sqrt{x^{2}+y^{2}}}\right)\right]\;\mathbf{j}+\\
 & \left[\dfrac{\partial\;}{\partial x}\left(\dfrac{x}{\sqrt{x^{2}+y^{2}}}\right)-\dfrac{\partial\;}{\partial y}\left(\dfrac{-y}{\sqrt{x^{2}+y^{2}}}\right)\right]\;\mathbf{k}\\
\\
= & 0\,\mathbf{i}-0\,\mathbf{j}+\left(\dfrac{\left(x^{2}+y^{2}\right)^{\frac{1}{2}}-x^{2}\left(x^{2}+y^{2}\right)^{-\frac{1}{2}}}{x^{2}+y^{2}}+\dfrac{\left(x^{2}+y^{2}\right)^{\frac{1}{2}}-y^{2}\left(x^{2}+y^{2}\right)^{-\frac{1}{2}}}{x^{2}+y^{2}}\right)\,\mathbf{k}\\
= & 0\,\mathbf{i}-0\,\mathbf{j}-\left[\dfrac{x^{2}+y^{2}-x^{2}+x^{2}+y^{2}-y^{2}}{\left(x^{2}+y^{2}\right)^{3/2}}\right]\,\mathbf{k}\\
= & 0\,\mathbf{i}-0\,\mathbf{j}-\dfrac{y^{2}+x^{2}}{\left(x^{2}+y^{2}\right)^{3/2}}\,\mathbf{k}\\
= & 0\,\mathbf{i}-0\,\mathbf{j}-\dfrac{1}{\sqrt{x^{2}+y^{2}}}\,\mathbf{k}
\end{align*}


Calcule o campo rotacional de $\mathbf{F}=\dfrac{-y\,\mathbf{i}+x\,\mathbf{j}}{x^{2}+y^{2}}$

\begin{align*}
\vec{\nabla}\times\mathbf{F}= & \begin{bmatrix}\mathbf{i} & \mathbf{j} & \mathbf{k}\\
\dfrac{\partial\;}{\partial x} & \dfrac{\partial\;}{\partial y} & \dfrac{\partial\;}{\partial z}\\
\dfrac{-y}{x^{2}+y^{2}} & \dfrac{x}{x^{2}+y^{2}} & 0
\end{bmatrix}\\
\\
= & \left[\dfrac{\partial\;}{\partial y}0-\dfrac{\partial\;}{\partial z}\left(\dfrac{x}{x^{2}+y^{2}}\right)\right]\;\mathbf{i}-\left[\dfrac{\partial\;}{\partial x}0-\dfrac{\partial\;}{\partial z}\left(\dfrac{-y}{x^{2}+y^{2}}\right)\right]\;\mathbf{j}+\\
 & \left[\dfrac{\partial\;}{\partial x}\left(\dfrac{x}{x^{2}+y^{2}}\right)-\dfrac{\partial\;}{\partial y}\left(\dfrac{-y}{x^{2}+y^{2}}\right)\right]\;\mathbf{k}\\
\\
= & 0\,\mathbf{i}-0\,\mathbf{j}+\left[\dfrac{\left(x^{2}+y^{2}\right)-2x^{2}}{\left(x^{2}+y^{2}\right)^{2}}+\dfrac{\left(x^{2}+y^{2}\right)-2y^{2}}{\left(x^{2}+y^{2}\right)^{2}}\right]\,\mathbf{k}\\
= & 0\,\mathbf{i}-0\,\mathbf{j}-\left(\dfrac{-x^{2}+y^{2}+x^{2}-y^{2}}{\left(x^{2}+y^{2}\right)^{2}}\right)\,\mathbf{k}\\
= & 0\,\mathbf{i}-0\,\mathbf{j}-0\,\mathbf{k}
\end{align*}


Calcule o campo rotacional de $\mathbf{F}=\dfrac{x\,\mathbf{i}+-y\,\mathbf{j}}{\sqrt{x^{2}+y^{2}}}$

\begin{align*}
\vec{\nabla}\times\mathbf{F}= & \begin{bmatrix}\mathbf{i} & \mathbf{j} & \mathbf{k}\\
\dfrac{\partial\;}{\partial x} & \dfrac{\partial\;}{\partial y} & \dfrac{\partial\;}{\partial z}\\
\dfrac{x}{\sqrt{x^{2}+y^{2}}} & \dfrac{-y}{\sqrt{x^{2}+y^{2}}} & 0
\end{bmatrix}\\
\\
= & \left[\dfrac{\partial\;}{\partial y}0-\dfrac{\partial\;}{\partial z}\left(\dfrac{-y}{\sqrt{x^{2}+y^{2}}}\right)\right]\;\mathbf{i}-\left[\dfrac{\partial\;}{\partial x}0-\dfrac{\partial\;}{\partial z}\left(\dfrac{x}{\sqrt{x^{2}+y^{2}}}\right)\right]\;\mathbf{j}+\\
 & \left[\dfrac{\partial\;}{\partial x}\left(\dfrac{-y}{\sqrt{x^{2}+y^{2}}}\right)-\dfrac{\partial\;}{\partial y}\left(\dfrac{x}{\sqrt{x^{2}+y^{2}}}\right)\right]\;\mathbf{k}\\
\\
= & 0\,\mathbf{i}-0\,\mathbf{j}+\left(\dfrac{yx\left(x^{2}+y^{2}\right)^{-\frac{1}{2}}}{x^{2}+y^{2}}-\dfrac{-xy\left(x^{2}+y^{2}\right)^{-\frac{1}{2}}}{x^{2}+y^{2}}\right)\,\mathbf{k}\\
= & 0\,\mathbf{i}-0\,\mathbf{j}-\left[\dfrac{2xy}{\left(x^{2}+y^{2}\right)^{3/2}}\right]\,\mathbf{k}
\end{align*}


Um escoamento é representado pelo campo de velocidades ${\displaystyle \mathbf{v}=10x\,\mathbf{i}-10y\,\mathbf{j}+30\,\mathbf{k}}$,
verifique (a) Se há um escoamento incompressível. (b) Se há um escoamento
irrotacional.

Para termos um escoamento incompressível devemos verificar que a divergência
do campo seja nula,

\begin{align*}
\vec{\nabla}\cdot\vec{v}= & \left(\dfrac{\partial\;}{\partial x}\,\mathbf{i}+\dfrac{\partial\;}{\partial y}\,\mathbf{j}+\dfrac{\partial\;}{\partial z}\,\mathbf{k}\right)\cdot\left(10x\,\mathbf{i}-10y\,\mathbf{j}+30\,\mathbf{k}\right)\\
= & \dfrac{\partial\;}{\partial x}\left(10x\right)+\dfrac{\partial\;}{\partial y}\left(-10y\right)+\dfrac{\partial\;}{\partial z}\left(30\right)\\
= & 10-10\\
= & 0
\end{align*}
por tanto temos um escoamento incompressível.

Agora devemos verificar que o rotacional do campo é nulo,

\begin{align*}
\vec{\nabla}\times\mathbf{F}= & \begin{bmatrix}\mathbf{i} & \mathbf{j} & \mathbf{k}\\
\dfrac{\partial\;}{\partial x} & \dfrac{\partial\;}{\partial y} & \dfrac{\partial\;}{\partial z}\\
10x & -10y & 30
\end{bmatrix}\\
= & \left[\dfrac{\partial\;}{\partial y}\left(30\right)-\dfrac{\partial\;}{\partial z}\left(10y\right)\right]\;\mathbf{i}-\left[\dfrac{\partial\;}{\partial x}\left(30\right)-\dfrac{\partial\;}{\partial z}\left(10x\right)\right]\;\mathbf{j}+\\
= & \left[\dfrac{\partial\;}{\partial x}\left(10y\right)-\dfrac{\partial\;}{\partial z}\left(10x\right)\right]\;\mathbf{k}\\
= & \vec{0}
\end{align*}
assim, irrotacional.

Para um escoamento no plano $xy$, a componente $y$ da velocidade
é dado por $v_{y}=y^{2}-2x+2y$, determine uma possível componente
em $x$ para um escoamento incompressível.

Sabemos que para que um escoamento seja incompressível é necessário
que a divergência do campo seja nula, assim
\begin{align*}
\vec{\nabla}\cdot\vec{v}= & \left(\dfrac{\partial\;}{\partial x}\,\mathbf{i}+\dfrac{\partial\;}{\partial y}\,\mathbf{j}\right)\cdot\left[v_{x}\,\mathbf{i}+\left(y^{2}-2x+2y\right)\,\mathbf{j}\right]\\
0= & \dfrac{\partial\;}{\partial x}v_{x}+\dfrac{\partial\;}{\partial y}\left(y^{2}-2x+2y\right)\\
0= & \dfrac{\partial v_{x}}{\partial x}+2y+2\\
\dfrac{\partial v_{x}}{\partial x}= & -2y-2
\end{align*}
pelo I teorema fundamental do cálculo
\begin{align*}
v_{x}= & \int\left(-2y-2\right)\, dx\\
= & -2yx-2x+C_{1}(y)
\end{align*}
onde $C_{1}(y)$ é uma contante que pode depender de $y$ (é constante
em relação à $x$), note que se derivamos a última expressão em relação
a $x$ reobtemos a expressão original.


\subsection*{Exemplo 06}

O campo eletrostático associado a uma carga positiva \$Q\$ é dado
pela equação $\mathbf{E}=\vec{\nabla}V$, onde ${\displaystyle V=\dfrac{Q}{r}}$,
sendo ${\displaystyle r=\sqrt{x^{2}+y^{2}}}$, verificar que o campo
$\mathbf{E}$ é irrotacional.

\begin{align*}
\mathbf{E}= & \vec{-\nabla}\, V\\
= & -\left(\dfrac{\partial\;}{\partial x}\,\mathbf{i}+\dfrac{\partial\;}{\partial y}\,\mathbf{j}\right)V\\
= & -\dfrac{\partial V}{\partial x}\,\mathbf{i}-\dfrac{\partial V}{\partial y}\,\mathbf{j}\\
= & -\dfrac{\partial\;}{\partial x}\left(\dfrac{Q}{\sqrt{x^{2}+y^{2}}}\right)\,\mathbf{i}-\dfrac{\partial\;}{\partial y}\left(\dfrac{Q}{\sqrt{x^{2}+y^{2}}}\right)\,\mathbf{j}\\
= & \dfrac{\frac{1}{2}Q\left(x^{2}+y^{2}\right)^{-\frac{1}{2}}\left(2x\right)}{x^{2}+y^{2}}\mathbf{\, i}+\dfrac{\frac{1}{2}Q\left(x^{2}+y^{2}\right)^{-\frac{1}{2}}\left(2y\right)}{x^{2}+y^{2}}\mathbf{\, j}\\
= & \dfrac{Qx}{\left(x^{2}+y^{2}\right)^{\frac{3}{2}}}\mathbf{\, i}+\dfrac{Qy}{\left(x^{2}+y^{2}\right)^{\frac{3}{2}}}\mathbf{\, j}\\
= & \dfrac{Q}{\left(x^{2}+y^{2}\right)^{\frac{3}{2}}}\left(x\mathbf{\, i}+y\mathbf{\, j}\right)\\
= & \dfrac{Q}{x^{2}+y^{2}}\left(\dfrac{x\mathbf{\, i}+y\mathbf{\, j}}{\sqrt{x^{2}+y^{2}}}\right)\\
= & \dfrac{Q}{x^{2}+y^{2}}\hat{r}
\end{align*}
dessa forma o rotacional está dado por

\begin{align*}
\vec{\nabla}\times\mathbf{F}= & \begin{bmatrix}\mathbf{i} & \mathbf{j} & \mathbf{k}\\
\dfrac{\partial\;}{\partial x} & \dfrac{\partial\;}{\partial y} & \dfrac{\partial\;}{\partial z}\\
\dfrac{Qx}{\left(x^{2}+y^{2}\right)^{\frac{3}{2}}} & \dfrac{Qy}{\left(x^{2}+y^{2}\right)^{\frac{3}{2}}} & 0
\end{bmatrix}\\
= & \left[\dfrac{\partial\;}{\partial y}\left(0\right)-\dfrac{\partial\;}{\partial z}\left(\dfrac{Qy}{\left(x^{2}+y^{2}\right)^{\frac{3}{2}}}\right)\right]\;\mathbf{i}-\left[\dfrac{\partial\;}{\partial x}\left(0\right)-\dfrac{\partial\;}{\partial z}\left(\dfrac{Qx}{\left(x^{2}+y^{2}\right)^{\frac{3}{2}}}\right)\right]\;\mathbf{j}+\\
= & \left[\dfrac{\partial\;}{\partial x}\left(\dfrac{Qy}{\left(x^{2}+y^{2}\right)^{\frac{3}{2}}}\right)-\dfrac{\partial\;}{\partial y}\left(\dfrac{Qx}{\left(x^{2}+y^{2}\right)^{\frac{3}{2}}}\right)\right]\;\mathbf{k}\\
= & 0\;\mathbf{i}+0\;\mathbf{j}+\left[-\dfrac{3Qyx}{\left(x^{2}+y^{2}\right)^{2}}+\dfrac{3Qyx}{\left(x^{2}+y^{2}\right)^{2}}\right]\;\mathbf{k}\\
= & \mathbf{0}
\end{align*}



\subsection*{Exemplo 07}

É o campo vetorial ${\displaystyle \mathbf{F}=2x^{2}y\,\mathbf{i}+5xz\,\mathbf{j}+x^{2}y^{2}\,\mathbf{k}}$
definido em $\mathbb{R}^{3}$ um campo conservativo?


\subsubsection*{Sol}

Para ser um campo conservativo deve verificar que o domínio é simplesmente
conexo e em se tratando de polinômios isto é verdade e além disso
o seu rotacional deve ser nulo:

\begin{align*}
\vec{\nabla}\times\mathbf{F}= & \begin{bmatrix}\mathbf{i} & \mathbf{j} & \mathbf{k}\\
\dfrac{\partial\;}{\partial x} & \dfrac{\partial\;}{\partial y} & \dfrac{\partial\;}{\partial z}\\
2x^{2}y & 5xz & x^{2}y^{2}
\end{bmatrix}\\
= & \left[\dfrac{\partial\;}{\partial y}\left(x^{2}y^{2}\right)-\dfrac{\partial\;}{\partial z}\left(5xz\right)\right]\;\mathbf{i}-\left[\dfrac{\partial\;}{\partial x}\left(x^{2}y^{2}\right)-\dfrac{\partial\;}{\partial z}\left(2x^{2}\right)\right]\;\mathbf{j}+\\
= & \left[\dfrac{\partial\;}{\partial x}\left(5xz\right)-\dfrac{\partial\;}{\partial y}\left(2x^{2}\right)\right]\;\mathbf{k}\\
= & \left(2x^{2}y-5x\right)\;\mathbf{i}+2xy^{2}\;\mathbf{j}+5z\;\mathbf{k}
\end{align*}
por tanto não é um campo conservativo.


\subsection*{Exemplo 08}

É o campo vetorial ${\displaystyle \mathbf{F}=\left(4xy+z\right)\,\mathbf{i}+2x^{2}\,\mathbf{j}+x\,\mathbf{k}}$
definido em $\mathbb{R}^{3}$ um campo conservativo?


\subsubsection*{Sol}

Para ser um campo conservativo deve verificar que o domínio é simplesmente
conexo e em se tratando de polinômios isto é verdade e além disso
o seu rotacional deve ser nulo:

\begin{align*}
\vec{\nabla}\times\mathbf{F}= & \begin{bmatrix}\mathbf{i} & \mathbf{j} & \mathbf{k}\\
\dfrac{\partial\;}{\partial x} & \dfrac{\partial\;}{\partial y} & \dfrac{\partial\;}{\partial z}\\
4xy+z & 2x^{2} & x
\end{bmatrix}\\
= & \left[\dfrac{\partial\;}{\partial y}\left(x\right)-\dfrac{\partial\;}{\partial z}\left(2x^{2}\right)\right]\;\mathbf{i}-\left[\dfrac{\partial\;}{\partial x}\left(x\right)-\dfrac{\partial\;}{\partial z}\left(4xy+z\right)\right]\;\mathbf{j}+\\
= & \left[\dfrac{\partial\;}{\partial x}\left(2x^{2}\right)-\dfrac{\partial\;}{\partial y}\left(4xy+z\right)\right]\;\mathbf{k}\\
= & 0\;\mathbf{i}+0\;\mathbf{j}+0\;\mathbf{k}
\end{align*}
por tanto é um campo conservativo.

\begin{align*}
(x-3)^{2}+y^{2} & <1\\
\end{align*}



\subsection*{Exemplo 12}

$\mathbf{F}(x,y,z)=\left(yz+2\right)\,\mathbf{i}+\left(xz+1\right)\,\mathbf{j}+\left(xy+2z\right)\,\mathbf{k}$

Saberemos que $\mathbf{F}$ é um campo gradiente se verificar $\vec{\nabla}\times\mathbf{F}=0$,
por tanto
\begin{align*}
\begin{bmatrix}\mathbf{i} & \mathbf{j} & \mathbf{k}\\
\dfrac{\partial\;}{\partial x} & \dfrac{\partial\;}{\partial y} & \dfrac{\partial\;}{\partial z}\\
F_{x} & F_{y} & F_{z}
\end{bmatrix} & =0\\
\left[\dfrac{\partial F_{z}}{\partial y}-\dfrac{\partial F_{y}}{\partial z}\right]\;\mathbf{i}-\left[\dfrac{\partial F_{z}}{\partial x}-\dfrac{\partial F_{x}}{\partial z}\right]\;\mathbf{j}+\left[\dfrac{\partial F_{y}}{\partial x}-\dfrac{\partial F_{x}}{\partial y}\right]\;\mathbf{k} & =0
\end{align*}
de onde
\[
\begin{array}{r}
\dfrac{\partial F_{z}}{\partial y}=\dfrac{\partial F_{y}}{\partial z}\\
\dfrac{\partial F_{z}}{\partial x}=\dfrac{\partial F_{x}}{\partial z}\\
\dfrac{\partial F_{y}}{\partial x}=\dfrac{\partial F_{x}}{\partial y}
\end{array}
\]
assim, então
\[
\begin{array}{lcrcc}
\dfrac{\partial\;}{\partial y}\left(xy+2z\right) & = & \dfrac{\partial\;}{\partial z}\left(xz+1\right) & = & x\\
\dfrac{\partial\;}{\partial x}\left(xy+2z\right) & = & \dfrac{\partial\;}{\partial z}\left(yz+2\right) & = & y\\
\dfrac{\partial\;}{\partial x}\left(xz+1\right) & = & \dfrac{\partial\;}{\partial y}\left(yz+2\right) & = & z
\end{array}
\]
dessa forma, a função vetorial $\mathbf{F}(x,y,z)$ deriva de um gradiente,
isto é

\[
\mathbf{F}(x,y,z)=\vec{\nabla f}
\]
o que implica
\begin{align*}
\dfrac{\partial f}{\partial x}= & F_{x}\\
\dfrac{\partial f}{\partial y}= & F_{y}\\
\dfrac{\partial f}{\partial z}= & F_{z}
\end{align*}
escolhamos a primeira dessa equações para resolver, pelo I teorema
fundamental do cálculo
\begin{align*}
f= & \int F_{x}dx\\
= & \int\left(yz+2\right)\, dx\\
= & xyz+2x+C_{1}(y,z)
\end{align*}
derivando a expressão anterior
\[
\dfrac{\partial f}{\partial y}=xz+\dfrac{\partial C_{1}}{\partial y}=F_{y}
\]
obtemos
\begin{align*}
xz+\dfrac{\partial C_{1}}{\partial y}= & xz+1\\
\dfrac{\partial C_{1}}{\partial y}= & 1\\
C_{1}= & \int dy\\
= & y+C_{2}(z)
\end{align*}
de forma que
\[
f=xyz+2x+y+C_{2}(z)
\]
derivando
\[
\dfrac{\partial f}{\partial z}=xy+\dfrac{\partial C_{2}}{\partial z}=F_{z}
\]
substituindo $F_{z}$
\begin{align*}
xy+\dfrac{\partial C_{2}}{\partial z}= & xy+2z\\
\dfrac{\partial C_{2}}{\partial z}= & 2z\\
C_{2}= & 2\int z\, dz\\
= & z^{2}+C
\end{align*}
assim
\[
f=xyz+2x+y+z^{2}+C
\]



\subsection*{Exemplo 13}

Considere o campo vetorial $\mathbf{F}$ em $\mathbb{R}^{3}$ definido
por 
\[
\mathbf{F}(x,y,z)=y\,\mathbf{i}+\left[z\cos\left(yz\right)+x\right]\,\mathbf{j}+\left[y\cos\left(yz\right)\right]\,\mathbf{k}
\]
mostre que $\mathbf{F}$ é irrotacional e encontre um potencial escalar


\subsubsection*{Sol}

\begin{align*}
\vec{\nabla}\times\mathbf{F}= & \begin{bmatrix}\mathbf{i} & \mathbf{j} & \mathbf{k}\\
\dfrac{\partial\;}{\partial x} & \dfrac{\partial\;}{\partial y} & \dfrac{\partial\;}{\partial z}\\
y & z\cos\left(yz\right)+x & y\cos\left(yz\right)
\end{bmatrix}\\
= & \left[\dfrac{\partial\;}{\partial y}\left(y\cos\left(yz\right)+x\right)-\dfrac{\partial\;}{\partial z}\left(z\cos\left(yz\right)\right)\right]\;\mathbf{i}-\left[\dfrac{\partial\;}{\partial x}\left(y\cos\left(yz\right)+x\right)-\dfrac{\partial\;}{\partial z}\left(y\right)\right]\;\mathbf{j}+\\
= & \left[\dfrac{\partial\;}{\partial x}\left(z\cos\left(yz\right)\right)-\dfrac{\partial\;}{\partial y}\left(y\right)\right]\;\mathbf{k}\\
= & \left[z\cos\left(yz\right)-z\cos\left(yz\right)\right]\;\mathbf{i}+\left[0-0\right]\;\mathbf{j}+\left[1-1\right]\;\mathbf{k}\\
= & 0
\end{align*}
por tanto deriva de um gradiente. Assim, calculamos o campo escalar
\begin{align*}
f= & \int y\, dx\\
= & xy+C_{2}(x,y)
\end{align*}


\begin{align*}
\dfrac{\partial f}{\partial y} & =x+\dfrac{\partial C_{2}}{\partial y}
\end{align*}
de onde

\begin{align*}
x+\dfrac{\partial C_{2}}{\partial y}= & z\cos\left(yz\right)+x\\
C_{2}= & \int z\,\cos\left(yz\right)\, dy\\
= & \sin\left(yz\right)+C_{1}(z)
\end{align*}
de forma
\[
f=xy+\sin\left(yz\right)+C_{1}(z)
\]
derivando
\[
\dfrac{\partial f}{\partial z}=y\cos\left(yz\right)+\dfrac{\partial C_{1}}{\partial z}=y\cos\left(yz\right)
\]
de onde
\[
\dfrac{\partial C_{1}}{\partial z}=0\Rightarrow C_{1}=C
\]
dessa forma
\[
f=xy+\sin\left(yz\right)+C
\]



\subsection*{Exemplo 14}

Podemos reescrever a equação com
\[
\mathbf{F}(x,y,z)=-\dfrac{GMm\mathbf{r}}{r^{3}}=-GMm\left(\dfrac{1}{r^{3}}\right)\mathbf{r}
\]
aplicando o rotacional
\begin{align*}
\vec{\nabla}\times\mathbf{F}= & \vec{\nabla}\times\left[-GMm\left(\dfrac{1}{r^{3}}\right)\mathbf{r}\right]\\
= & -GMm\left[\vec{\nabla}\left(\dfrac{1}{r^{3}}\right)\times\mathbf{r}+\left(\dfrac{1}{r^{3}}\right)\vec{\nabla}\times\mathbf{r}\right]
\end{align*}
dessa forma devemos calcular o gradiente

\begin{align*}
\vec{\nabla}\left(\dfrac{1}{r^{3}}\right)= & \left[\dfrac{\partial\;}{\partial x}\;\mathbf{i}+\dfrac{\partial\;}{\partial y}\;\mathbf{j}+\dfrac{\partial\;}{\partial z}\;\mathbf{k}\right]\left(\dfrac{1}{\left(x^{2}+y^{2}+z^{2}\right)^{3/2}}\right)\\
= & \dfrac{3x\left(x^{2}+y^{2}+z^{2}\right)^{1/2}}{\left(x^{2}+y^{2}+z^{2}\right)^{3}}\;\mathbf{i}+\dfrac{3y\left(x^{2}+y^{2}+z^{2}\right)^{1/2}}{\left(x^{2}+y^{2}+z^{2}\right)^{3}}\;\mathbf{j}+\dfrac{3z\left(x^{2}+y^{2}+z^{2}\right)^{1/2}}{\left(x^{2}+y^{2}+z^{2}\right)^{3}}\;\mathbf{k}\\
= & \dfrac{3}{\left(x^{2}+y^{2}+z^{2}\right)^{5/2}}\left(x\;\mathbf{i}+y\;\mathbf{j}+z\;\mathbf{k}\right)
\end{align*}


\begin{align*}
\vec{\nabla}\left(\dfrac{1}{r^{3}}\right)\times\mathbf{r}= & \begin{bmatrix}\mathbf{i} & \mathbf{j} & \mathbf{k}\\
\dfrac{3x}{\left(x^{2}+y^{2}+z^{2}\right)^{5/2}} & \dfrac{3y}{\left(x^{2}+y^{2}+z^{2}\right)^{5/2}} & \dfrac{3z}{\left(x^{2}+y^{2}+z^{2}\right)^{5/2}}\\
x & y & z
\end{bmatrix}\\
= & \dfrac{3}{\left(x^{2}+y^{2}+z^{2}\right)^{5/2}}\left[\left(yz-yz\right)\;\mathbf{i}+\left(xz-xz\right)\;\mathbf{j}+\left(xy-xy\right)\;\mathbf{k}\right]\\
= & \vec{0}
\end{align*}


\begin{align*}
\vec{\nabla}\times\mathbf{r}= & \begin{bmatrix}\mathbf{i} & \mathbf{j} & \mathbf{k}\\
\dfrac{\partial\;}{\partial x} & \dfrac{\partial\;}{\partial y} & \dfrac{\partial\;}{\partial z}\\
x & y & z
\end{bmatrix}\\
= & \left(0-0\right)\;\mathbf{i}-\left(0-0\right)\;\mathbf{j}+\left(0-0\right)\;\mathbf{k}\\
= & \vec{0}
\end{align*}
dessa forma
\[
\vec{\nabla}\times\mathbf{F}=\vec{0}
\]

\end{document}
