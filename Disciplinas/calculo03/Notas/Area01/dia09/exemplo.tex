%% LyX 2.0.6 created this file.  For more info, see http://www.lyx.org/.
%% Do not edit unless you really know what you are doing.
\documentclass[12pt,english,brazil,a4papper]{article}
\usepackage[T1]{fontenc}
\usepackage[utf8x]{inputenc}
\usepackage{geometry}
\geometry{verbose,tmargin=2cm,bmargin=2cm,lmargin=2cm,rmargin=2cm}
\usepackage{amsmath}
\usepackage{amssymb}
\usepackage{mathdots}
\usepackage{setspace}
\usepackage{esint}
\onehalfspacing

\makeatletter
%%%%%%%%%%%%%%%%%%%%%%%%%%%%%% User specified LaTeX commands.

\usepackage{babel}
\usepackage{amsfonts}



 

\makeatother

\usepackage{babel}
\begin{document}

\subsection*{Exemplo 01}

Calcule a integral $\int_{C}\mathbf{F}\cdot d\mathbf{r}$, onde $\mathbf{F}=z\,\mathbf{i}+xy\,\mathbf{j}-y^{2}\,\mathbf{k}$
ao longo da curva dada por $\mathbf{r}(t)=t^{2}\,\mathbf{i}+t\,\mathbf{j}+\sqrt{t}\,\mathbf{k}$,
$0\leq t\leq1$


\subsubsection*{SOL}

Temos
\[
\mathbf{F}\left(\mathbf{r}(t)\right)=\sqrt{t}\,\mathbf{i}+t^{3}\,\mathbf{j}-t^{2}\,\mathbf{k}
\]
e
\[
\mathbf{v}(t)=2t\,\mathbf{i}+\mathbf{j}+\dfrac{1}{2\sqrt{t}}\,\mathbf{k}
\]
de forma que
\begin{align*}
\int_{C}\mathbf{F}\left(\mathbf{r}(t)\right)\cdot d\mathbf{r}= & \int_{0}^{1}\mathbf{F}\left(\mathbf{r}(t)\right)\cdot\mathbf{v}(t)dt\\
= & \int_{0}^{1}\left(2t^{3/2}+t^{3}-\dfrac{1}{2}t^{3/4}\right)dt\\
= & \left[\left(\dfrac{2}{3}\right)\left(\dfrac{2}{5}t^{5/2}\right)+\dfrac{1}{4}t^{4}\right]_{0}^{1}\\
= & \dfrac{17}{20}
\end{align*}



\subsection*{Exemplo 02}

Calcule a integral $\int_{C}\mathbf{F}\cdot d\mathbf{r}$, onde $\mathbf{F}=z\,\mathbf{i}+y^{2}\,\mathbf{j}-x\,\mathbf{k}$
ao longo da curva dada por $\mathbf{r}(t)=\left(t+1\right)\,\mathbf{i}+e^{t}\,\mathbf{j}+t^{2}\,\mathbf{k}$,
$0\leq t\leq2$


\subsubsection*{SOL}

Temos
\[
\mathbf{F}\left(\mathbf{r}(t)\right)=t^{2}\,\mathbf{i}+e^{2t}\,\mathbf{j}-\left(t+1\right)\,\mathbf{k}
\]
e

\[
\mathbf{v}(t)=\mathbf{i}+e^{t}\,\mathbf{j}+2t\,\mathbf{k}
\]
de forma que
\begin{align*}
\int_{C}\mathbf{F}\left(\mathbf{r}(t)\right)\cdot d\mathbf{r}= & \int_{0}^{2}\mathbf{F}\left(\mathbf{r}(t)\right)\cdot\mathbf{v}(t)dt\\
= & \int_{0}^{2}\left(e^{3t}+3t^{2}+2t\right)dt\\
= & \left[\dfrac{1}{3}d^{3t}+t^{3}+t^{2}\right]_{0}^{2}\\
= & \left(\dfrac{1}{3}e^{6}+8+4\right)-\dfrac{1}{3}\\
= & \dfrac{1}{3}\left(e^{6}+35\right)
\end{align*}



\subsection*{Exemplo 03}

Seja $\mathbf{r}(t)=\sin t\,\mathbf{i}+\cos t\,\mathbf{j}+t\,\mathbf{k}$
com $0\leq t\leq2\pi$. Seja o campo vetorial definido por $\mathbf{F}=x\,\mathbf{i}+y\,\mathbf{j}+z\,\mathbf{k}$.
Calcule $\int_{C}\mathbf{F}\cdot d\mathbf{r}$


\subsubsection*{SOL}

Temos
\[
\mathbf{F}\left(\mathbf{r}(t)\right)=\sin t\,\mathbf{i}+\cos t\,\mathbf{j}+t\,\mathbf{k}
\]
e

\[
\mathbf{v}(t)=-\cos t\,\mathbf{i}+\sin t\,\mathbf{j}+\,\mathbf{k}
\]
de forma que
\begin{align*}
\int_{C}\mathbf{F}\left(\mathbf{r}(t)\right)\cdot d\mathbf{r}= & \int_{0}^{2\pi}\mathbf{F}\left(\mathbf{r}(t)\right)\cdot\mathbf{v}(t)dt\\
= & \int_{0}^{2\pi}\left(-\cos t\,\sin t+\cos t\,\sin t+t\right)dt\\
= & \left[\dfrac{1}{2}t^{2}\right]_{0}^{2\pi}\\
= & 2\pi^{2}
\end{align*}



\subsection*{Exemplo 04}

Seja $\mathbf{r}(t)=\cos^{3}t\,\mathbf{i}+\sin t\,\mathbf{j}+t\,\mathbf{k}$,
onde $0\leq t\leq\dfrac{7}{2}\pi$, como se mostra na figura embaixo.
Calcule a integral de linha $\int_{C}\left(\sin z\, dx+\cos z\, dy-\left(xy\right)^{1/3}dz\right)$


\subsubsection*{SOL}

Para este caso temos
\begin{align*}
\dfrac{dx}{dt}= & -3\cos^{2}t\,\sin t\\
\dfrac{dy}{dt}= & 3\cos^{2}t\,\cos t\\
\dfrac{dz}{dt}= & 1
\end{align*}
tal que a integral é
\begin{align*}
\int_{C}\left(\sin z\, dx+\cos z\, dy-\left(xy\right)^{1/3}dz\right)= & -\int_{0}^{\frac{7\pi}{2}}3\sin t\,\cos^{2}t\,\sin t\, dt+\int_{0}^{\frac{7\pi}{2}}3\sin t\,\cos^{2}t\,\cos t\, dt-\int_{0}^{\frac{7\pi}{2}}\cos t\,\sin\, t\, dt\\
= & -\int_{0}^{\frac{7\pi}{2}}\cos t\,\sin\, t\, dt\\
= & -\left[\sin^{2}t\right]_{0}^{\frac{7\pi}{2}}\\
= & -\dfrac{1}{2}
\end{align*}



\subsection*{Exemplo 05}

\selectlanguage{english}%
$\,\mathbf{i}+\,\mathbf{j}+\,\mathbf{k}$

\selectlanguage{brazil}%
Uma elipse orientada em sentido anti-horário é tem por parametrização
\foreignlanguage{english}{$\mathbf{r}(t)=\left(5+4\cos t\right)\,\mathbf{i}+\left(3+2\cos t\right)\,\mathbf{j}$}
onde $0\leq t\leq2\pi,$ calcule $\int_{C}2y\, dx-3\, dy$


\subsubsection*{SOL}

\begin{align*}
\dfrac{dx}{dt}= & -4\sin t\\
\dfrac{dy}{dt}= & 2\cos t
\end{align*}
assim
\begin{align*}
\int_{C}2y\, dx-3\, dy= & 2\int_{0}^{2\pi}\left(3+2\sin t\right)\left(-4\sin t\right)\, dt-3\int_{0}^{2\pi}\left(2\cos t\right)\, dt\\
= & -24\int_{0}^{2\pi}\sin t\, dt-16\int_{0}^{2\pi}\sin^{2}t\, dt-3\int_{0}^{2\pi}\left(2\cos t\right)\, dt\\
= & -16\int_{0}^{2\pi}\sin^{2}t\, dt\\
= & -16\pi
\end{align*}



\subsection*{Exemplo 06}

Seja $\mathbf{F}(x,y)=y^{2}\,\mathbf{i}+2xy\,\mathbf{j}$. Avalie
a integral de linha de $\left(0,\,0\right)$ até $\left(1,\,1\right)$
seguindo 

1) A linha reta $y=x$

2) A curva $y=x^{2}$

3) A curva suave definida por partes que consiste dos segmentos de
reta que vão desde $\left(0,\,0\right)$ até $\left(0,\,1\right)$
e desde $\left(0,\,1\right)$ até $\left(1,\,1\right)$.

Demonstre que o campo é um campo gradiente e determine o campo potencial.

\selectlanguage{english}%
$\,\mathbf{i}+\,\mathbf{j}+\,\mathbf{k}$

\selectlanguage{brazil}%

\subsubsection*{SOL}

Seguindo o caminho (1):

\begin{align*}
\mathbf{r}(t)= & t\,\mathbf{i}+t\,\mathbf{j}\\
\mathbf{v}(t)= & \mathbf{i}+\mathbf{j}
\end{align*}
onde $0\leq t\leq1$, dessa forma a 
\begin{align*}
\mathbf{F}\cdot d\mathbf{r}= & \left(y^{2}\,\mathbf{i}+2xy\,\mathbf{j}\right)\cdot\left(\mathbf{i}+\mathbf{j}\right)\\
= & \left(t^{2}\,\mathbf{i}+2t^{2}\,\mathbf{j}\right)\cdot\left(\mathbf{i}+\mathbf{j}\right)\\
= & t^{2}+2t^{2}\\
= & 3t^{2}
\end{align*}
dessa forma
\begin{align*}
\int_{c}\mathbf{F}\cdot d\mathbf{r}= & 3\int_{0}^{1}t^{2}dt\\
= & \dfrac{3}{3}\left.t^{3}\right|_{0}^{1}\\
= & 1
\end{align*}


Seguindo o caminho (1):

\begin{align*}
\mathbf{r}(t)= & t\,\mathbf{i}+t^{2}\,\mathbf{j}\\
\mathbf{v}(t)= & \mathbf{i}+2t\,\mathbf{j}
\end{align*}
onde $0\leq t\leq1$, dessa forma a 
\begin{align*}
\mathbf{F}\cdot d\mathbf{r}= & \left(y^{2}\,\mathbf{i}+2xy\,\mathbf{j}\right)\cdot\left(\mathbf{i}+2t\,\mathbf{j}\right)\\
= & \left(t^{4}\,\mathbf{i}+2t^{3}\,\mathbf{j}\right)\cdot\left(\mathbf{i}+2t\,\mathbf{j}\right)\\
= & t^{4}+4t^{4}\\
= & 5t^{4}
\end{align*}
dessa forma
\begin{align*}
\int_{c}\mathbf{F}\cdot d\mathbf{r}= & 5\int_{0}^{1}t^{4}dt\\
= & \dfrac{5}{5}\left.t^{5}\right|_{0}^{1}\\
= & 1
\end{align*}


Seguindo o caminho (3): Neste caso devemos dividir a integral
\[
\int_{c3}\mathbf{F}\cdot d\mathbf{r}=\int_{c3_{a}}\mathbf{F}\cdot d\mathbf{r}+\int_{c3_{b}}\mathbf{F}\cdot d\mathbf{r}
\]


Seguindo o caminho (3a):

\begin{align*}
\mathbf{r}(t)= & \left(0,\,0\right)+\left[\left(0,\,1\right)-\left(0,\,0\right)\right]t\\
= & t\,\mathbf{j}\\
\mathbf{v}(t)= & \mathbf{j}
\end{align*}
onde $0\leq t\leq1$, dessa forma a 
\begin{align*}
\mathbf{F}\cdot d\mathbf{r}= & \left(y^{2}\,\mathbf{i}+2xy\,\mathbf{j}\right)\cdot\left(\mathbf{j}\right)\\
= & \left(t^{2}\,\mathbf{i}\right)\cdot\left(\mathbf{j}\right)\\
= & 0
\end{align*}
dessa forma
\begin{align*}
\int_{c3_{a}}\mathbf{F}\cdot d\mathbf{r}= & \int_{0}^{1}0\, dt\\
= & 0
\end{align*}


Seguindo o caminho (3b):

\begin{align*}
\mathbf{r}(t)= & \left(0,\,1\right)+\left[\left(1,\,1\right)-\left(0,\,1\right)\right]t\\
= & t\,\mathbf{i}+\,\mathbf{j}\\
\mathbf{v}(t)= & \mathbf{i}
\end{align*}
onde $0\leq t\leq1$, dessa forma a 
\begin{align*}
\mathbf{F}\cdot d\mathbf{r}= & \left(y^{2}\,\mathbf{i}+2xy\,\mathbf{j}\right)\cdot\left(\mathbf{i}\right)\\
= & \left(\mathbf{i}+2t\,\mathbf{j}\right)\cdot\left(\mathbf{i}\right)\\
= & 1
\end{align*}
dessa forma
\begin{align*}
\int_{c3_{b}}\mathbf{F}\cdot d\mathbf{r}= & \int_{0}^{1}dt\\
= & \left.t\right|_{0}^{1}\\
= & 1
\end{align*}
por tanto, 
\begin{align*}
\int_{c3}\mathbf{F}\cdot d\mathbf{r}= & \int_{c3_{a}}\mathbf{F}\cdot d\mathbf{r}+\int_{c3_{b}}\mathbf{F}\cdot d\mathbf{r}\\
= & 0+1\\
= & 1
\end{align*}
Observe que todas as soluções deram o mesmo resultado, por tanto a
integral ser independente do caminho escolhido. Isso é um indicativo
de que o campo é um campo gradiente, verificando
\begin{align*}
\vec{\nabla}\times\mathbf{F}= & \begin{bmatrix}\mathbf{i} & \mathbf{j} & \mathbf{k}\\
\dfrac{\partial\;}{\partial x} & \dfrac{\partial\;}{\partial y} & \dfrac{\partial\;}{\partial z}\\
y^{2} & 2xy & 0
\end{bmatrix}\\
= & \left(0-0\right)\;\mathbf{i}-\left(0-0\right)\;\mathbf{j}+\left(2y-2y\right)\;\mathbf{k}\\
= & \mathbf{0}
\end{align*}
assim
\begin{align*}
\mathbf{F}= & \vec{\nabla}f\\
F_{x}\,\mathbf{i}+F_{y}\,\mathbf{j}+F_{z}\,\mathbf{k}= & \dfrac{\partial f}{\partial x}\,\mathbf{i}+\dfrac{\partial f}{\partial y}\,\mathbf{j}+\dfrac{\partial f}{\partial z}\,\mathbf{k}\\
F_{x}= & \dfrac{\partial f}{\partial x}\\
F_{y}= & \dfrac{\partial f}{\partial y}\\
F_{z}= & \dfrac{\partial f}{\partial z}
\end{align*}
dessa forma
\begin{align*}
\dfrac{\partial f}{\partial x}= & y^{2}\\
f= & \int y^{2}dx\\
= & xy^{2}+C_{1}\left(y,\, z\right)
\end{align*}
derivando
\begin{align*}
\dfrac{\partial\,}{\partial y}\left(xy^{2}+C_{1}\left(y,\, z\right)\right)= & F_{y}\\
2xy+\dfrac{\partial C_{1}}{\partial y}= & 2xy\\
C_{1}= & \int0\, dy\\
= & C_{2}\left(z\right)
\end{align*}
dessa forma
\[
f=xy^{2}+C_{1}\left(z\right)
\]
que, derivando
\begin{align*}
\dfrac{\partial\,}{\partial z}\left(xy^{2}+C_{2}\left(z\right)\right)= & F_{z}\\
\dfrac{\partial C_{2}}{\partial z}= & 0\\
C_{2}= & const
\end{align*}
por tanto
\[
f=xy^{2}+C
\]
Como
\[
D\left[\mathbf{F}\right]=\mathbb{R}\Rightarrow\text{domínio conexo}
\]
dessa forma
\begin{align*}
\int_{c}\mathbf{F}\cdot d\mathbf{r}= & f(B)-f(A)\\
= & f(1,\,1)-f(0,\,0)\\
= & 1-0\\
= & 1
\end{align*}



\subsection*{Exemplo 07}

Avaliar a integral 
\[
\int_{C}y\, dx+x^{2}dy
\]
onde $C$ é o arco parabólico dado por $y=4x-x^{2}$ desde $(1,\,3)$
até $(4,\,0)$, como se mostra na figura. Seguidamente calcule a mesma
integral mas agora seguindo o caminho da curva com orientação oposta
a descrita pela equação anterior.

\selectlanguage{english}%
$\,\mathbf{i}+\,\mathbf{j}+\,\mathbf{k}$

\selectlanguage{brazil}%

\subsubsection*{SOL}

Primeiro calculamos para o caso da curva com orientação positiva:

\begin{align*}
\mathbf{r}(t)= & t\,\mathbf{i}+\left(4t-t^{2}\right)\,\mathbf{j}\\
\mathbf{v}(t)= & \mathbf{i}+\left(4-2t\right)\,\mathbf{j}
\end{align*}
então
\begin{align*}
\int_{C}y\, dx+x^{2}dy= & \int_{1}^{4}\left(4t-t^{2}\right)dt+\int_{1}^{4}t^{2}\left(4-2t\right)dt\\
= & \int_{1}^{4}\left(4t+3t^{2}-2t^{3}\right)dt\\
= & \left[2t^{2}+t^{3}-\dfrac{1}{2}t^{4}\right]_{1}^{4}\\
= & \left(32+64-128\right)-\left(2+1-\dfrac{1}{2}\right)\\
= & -\dfrac{69}{2}
\end{align*}
Para o caso da curva com orientação oposta:

\begin{align*}
\mathbf{r}^{-}(t)= & \mathbf{r}(1+4-t)\\
= & \left(5-t\right)\,\mathbf{i}+\left[4\left(5-t\right)-\left(5-t\right)^{2}\right]\,\mathbf{j}\\
= & \left(5-t\right)\,\mathbf{i}+\left[\left(5-t\right)\left(4-5+t\right)\right]\,\mathbf{j}\\
= & \left(5-t\right)\,\mathbf{i}+\left(5-t\right)\left(t-1\right)\,\mathbf{j}\\
= & \left(5-t\right)\,\mathbf{i}+\left(5t-5-t^{2}+t\right)\,\mathbf{j}\\
= & \left(5-t\right)\,\mathbf{i}+\left(-5+6t-t^{2}\right)\,\mathbf{j}\\
\mathbf{v}^{-}(t)= & -\mathbf{i}+\left(6-2t\right)\,\mathbf{j}
\end{align*}
de forma que
\begin{align*}
\int_{C}y\, dx+x^{2}dy= & -\int_{1}^{4}\left(-5+6t-t^{2}\right)dt+\int_{1}^{4}\left(5-t\right)^{2}\left(6-2t\right)dt\\
= & -\int_{1}^{4}\left(-5+6t-t^{2}\right)dt+\int_{1}^{4}\left(25-10t+t^{2}\right)\left(6-2t\right)dt\\
= & -\int_{1}^{4}\left(-5+6t-t^{2}\right)dt+\int_{1}^{4}\left(150-60t+6t^{2}-50t+20t^{2}-2t^{3}\right)dt\\
= & \int_{1}^{4}\left(5-6t+t^{2}\right)dt+\int_{1}^{4}\left(150-110t+26t^{2}-2t^{3}\right)dt\\
= & \int_{1}^{4}\left(155-116t+27t^{2}-2t^{3}\right)dt\\
= & \left[155t-58t^{2}+9t^{3}-\dfrac{1}{2}t^{4}\right]_{1}^{4}\\
= & \left(620-928+576-128\right)-\left(155-58+9-\dfrac{1}{2}\right)\\
= & \dfrac{69}{2}
\end{align*}



\subsection*{Exemplo 07}

Calcular o trabalho realizado ao mover uma partícula desde $P_{1}=\left(0,\,0,\,0\right)$
até $P_{2}=\left(4,\,8,\,1\right)$ ao longo do caminho
\[
\mathbf{r}(t)=\left(t^{2},\, t^{3},\, t\right)\quad\text{(em metros)}\qquad\qquad\text{para }1\leq t\leq2
\]
na presencia da força $\mathbf{F}(t)=\left(x^{2},\,-z,\,-\dfrac{y}{z}\right)$,
dada em newtons.


\subsubsection*{SOL}

Escrevendo a força em termos da posição
\begin{align*}
\mathbf{F}(\mathbf{r})= & \mathbf{F}\left(t^{2},\, t^{3},\, t\right)\\
= & \left(t^{4}\,-t,\,-t^{2}\right)
\end{align*}
agora calculando a velocidade
\[
\mathbf{v}(t)=\left(2t,\,3t^{2},\,1\right)
\]
de forma que
\begin{align*}
\mathbf{F}\cdot\mathbf{v}= & \left(t^{4}\,-t,\,-t^{2}\right)\cdot\left(2t,\,3t^{2},\,1\right)\\
= & 2t^{5}-3t^{3}-t^{2}
\end{align*}
assim
\begin{align*}
W= & \int_{C}\mathbf{F}\cdot d\mathbf{r}\\
= & \int_{C}\mathbf{F}\cdot\mathbf{v}\, dt\\
= & \int_{1}^{2}\left(2t^{5}-3t^{3}-t^{2}\right)\, dt\\
= & \left[\dfrac{1}{3}t^{6}-\dfrac{3}{4}t^{4}-\dfrac{1}{3}t^{3}\right]_{1}^{2}\\
= & \left(\dfrac{64}{3}-12-\dfrac{8}{3}\right)-\left(\dfrac{1}{3}-\dfrac{3}{4}-\dfrac{1}{3}\right)\\
= & \dfrac{20}{3}-\dfrac{3}{4}\\
= & -\dfrac{89}{12}
\end{align*}



\subsection*{Exemplo 08}

Encontre o trabalho feito pelo campo de força

\[
\mathbf{F}(x,y,z)=-\dfrac{1}{2}x\,\mathbf{i}-\dfrac{1}{2}y\,\mathbf{j}+\dfrac{1}{4}\,\mathbf{k}
\]
onde as partículas se movem ao longo da hélice dado por
\[
\mathbf{r}(t)=\cos t\,\mathbf{i}+\sin t\,\mathbf{j}+t\,\mathbf{k}
\]
desde o ponto $\left(1,\,0,\,0\right)$ até $\left(-1,\,0,\,3\pi\right)$,
como se mostra na figura.


\subsubsection*{SOL}

Escrevendo a força em termo da posição

\selectlanguage{english}%
\[
\mathbf{F}(\mathbf{r}(t))=-\dfrac{1}{2}\cos t\,\mathbf{i}-\dfrac{1}{2}\sin t\,\mathbf{j}+\dfrac{1}{4}\,\mathbf{k}
\]
\foreignlanguage{brazil}{escrevendo a velocidade 
\[
\mathbf{v}(t)=-\sin t\,\mathbf{i}+\cos t\,\mathbf{j}+\mathbf{k}
\]
Considerando que os pontos entre os que será feita a integração são
de $\left(1,\,0,\,0\right)$ até $\left(-1,\,0,\,3\pi\right)$ o que
representa $t=0$ até $t=3\pi$, temos que a integral será
\begin{align*}
W= & \int_{c}\mathbf{F}\cdot d\mathbf{r}\\
= & \int_{0}^{3\pi}\left(-\dfrac{1}{2}\cos t\,\mathbf{i}-\dfrac{1}{2}\sin t\,\mathbf{j}+\dfrac{1}{4}\,\mathbf{k}\right)\cdot\left(-\sin t\,\mathbf{i}+\cos t\,\mathbf{j}+\mathbf{k}\right)dt\\
= & \int_{0}^{3\pi}\left(\dfrac{1}{2}\sin t\cos t-\dfrac{1}{2}\sin t\cos t+\dfrac{1}{4}\right)dt\\
= & \dfrac{1}{4}\int_{0}^{3\pi}dt\\
= & \dfrac{3\pi}{4}
\end{align*}
}

\selectlanguage{brazil}%

\subsection*{Exemplo 09}

Calcule a integral o trabalho realizado pela força $\mathbf{F}(x,y)=-y\,\mathbf{i}+x\,\mathbf{j}$
ao longo das curvas $x^{2}+y^{2}=9$ (para $x,\, y$ no I quadrante)
e $\mathbf{r}(t)=t\,\mathbf{i}+2t\,\mathbf{j}$ (neste caso $0\leq t\leq1$).


\subsubsection*{SOL}

Para o caso do I caminho observamos que uma parametrização adequada
é
\[
\mathbf{r}(t)=3\cos t\,\mathbf{i}+3\sin t\,\mathbf{j},\qquad0\leq t\leq\dfrac{\pi}{2}
\]
de onde a velocidade está dada por
\[
\mathbf{v}(t)=-3\sin t\,\mathbf{i}+3\cos t\,\mathbf{j}
\]
nessa situação
\[
\mathbf{F}(\mathbf{r}(t))=-\sin t\,\mathbf{i}+\cos t\,\mathbf{j}
\]
note que a velocidade e a força são vetores paralelos
\[
\mathbf{v}(t)=3\mathbf{F}(\mathbf{r}(t))
\]
A integral de linha
\begin{align*}
W= & \int_{c}\mathbf{F}\cdot d\mathbf{r}\\
= & \int_{0}^{\frac{\pi}{2}}\left(-\sin t\,\mathbf{i}+\cos t\,\mathbf{j}\right)\cdot\left(-3\sin t\,\mathbf{i}+3\cos t\,\mathbf{j}\right)dt\\
= & 3\int_{0}^{\frac{\pi}{2}}dt\\
= & \dfrac{3\pi}{2}
\end{align*}
Para o caso do outro caminho
\[
\mathbf{F}(\mathbf{r}(t))=-2t\,\mathbf{i}+t\,\mathbf{j}
\]
enquanto a velocidade está dada por
\[
\mathbf{v}(t)=\mathbf{i}+2\,\mathbf{j}
\]
de onde
\begin{align*}
W= & \int_{c}\mathbf{F}\cdot d\mathbf{r}\\
= & \int_{0}^{1}\left(-2t\,\mathbf{i}+t\,\mathbf{j}\right)\cdot\left(\mathbf{i}+2\,\mathbf{j}\right)dt\\
= & \int_{0}^{1}\left(-2t+2t\right)dt\\
= & 0
\end{align*}
o que implica que o campo de forças é perpendicular à direção do movimento,
como mostra a figura.


\subsection*{Exemplo 10}

Seja $\mathbf{F}(x,y,z)=\left(2xy+z,\, x^{2},\, x\right)$, 
\begin{enumerate}
\item Verifique que a função $V(x,\, y,z)=x^{2}y+xz$ é uma função potencial
do campo
\item Avalie $\int_{c}\mathbf{F}\cdot d\mathbf{r}$, onde $\mathbf{r}$
é o caminho $P=\left(1,\,-1,2\right)$ até $Q=\left(2,\,2,\,3\right)$
\end{enumerate}
\selectlanguage{english}%
$\,\mathbf{i}+\,\mathbf{j}+\,\mathbf{k}$

\selectlanguage{brazil}%

\subsubsection*{SOL}

Calculando o gradiente do campo escalar
\begin{align*}
\vec{\nabla}V= & \dfrac{\partial V}{\partial x}\,\mathbf{i}+\dfrac{\partial V}{\partial y}\,\mathbf{j}+\dfrac{\partial V}{\partial z}\,\mathbf{k}\\
= & \left(2y+z\right)\,\mathbf{i}+x^{2}\,\mathbf{j}+x\,\mathbf{k}
\end{align*}
que exatamente igual ao campo $\mathbf{F}(x,y,z)$, assim
\begin{align*}
\int_{c}\mathbf{F}\cdot d\mathbf{r}= & V(Q)-V(P)\\
= & \left[2^{2}(2)+(3)(2)\right]-\left[1^{2}*(-1)+(1)(2)\right]\\
= & 13
\end{align*}



\subsection*{Exemplo 11}

Mostre que
\[
\mathbf{F}=\left\langle 2xy+y^{3},\, x^{2}+3xy^{2}+2y\right\rangle 
\]
é um campo conservativo e encontre a função potencial.


\subsubsection*{SOL}

Primeiro calculamos as derivadas cruzadas a fim de testar se elas
são iguais
\begin{align*}
\dfrac{\partial F_{x}}{\partial y}= & \dfrac{\partial\,}{\partial y}\left(2xy+y^{3}\right)\\
= & 2x+3y^{2}\\
\dfrac{\partial F_{y}}{\partial x}= & \dfrac{\partial\,}{\partial x}\left(x^{2}+3xy^{2}+2y\right)\\
= & 2x+3y^{2}
\end{align*}
Isto, junto ao fato de que $D\left[\mathbf{F}\right]=\mathbb{R}^{2}$,
o qual é um simplesmente conexo, nos permite afirmar que o campo define
um campo conservativo. A função potencial associada a esse campo conservativo
é
\begin{align*}
\dfrac{\partial f}{\partial x}= & F_{x}\\
f= & \int\left(2xy+y^{3}\right)\, dx\\
= & x^{2}y+xy^{3}+C_{1}(y)
\end{align*}
derivando em relação a $y$
\begin{align*}
F_{y}= & \dfrac{\partial f}{\partial y}\\
x^{2}+3xy^{2}+2y= & \dfrac{\partial\,}{\partial y}\left(x^{2}y+xy^{3}+C_{1}(y)\right)\\
x^{2}+3xy^{2}+2y= & x^{2}+3xy^{2}+\dfrac{\partial C_{1}}{\partial y}\\
2y= & \dfrac{\partial C_{1}}{\partial y}\\
C_{1}= & 2\int y\, dy\\
= & y+C
\end{align*}
por tanto
\[
f=x^{2}y+xy^{3}+y+C
\]



\subsection*{Exemplo 12}

Mostre que o campo ciclone 
\[
\mathbf{F}=\left\langle -\dfrac{y}{x^{2}+y^{2}},\,\dfrac{x}{x^{2}+y^{2}}\right\rangle 
\]
é um campo irrotacional porem não é um campo conservativo.


\subsubsection*{SOL}

Para ser um campo irrotacional as derivadas parciais cruzadas devem
ser iguis
\begin{align*}
\dfrac{\partial F_{x}}{\partial y}= & \dfrac{\partial\,}{\partial y}\left(-\dfrac{y}{x^{2}+y^{2}}\right)\\
= & \dfrac{-\left(x^{2}+y^{2}\right)+y\dfrac{\partial\,}{\partial y}\left(x^{2}+y^{2}\right)}{\left(x^{2}+y^{2}\right)^{2}}\\
= & \dfrac{y^{2}-x^{2}}{x^{2}+y^{2}}\\
\dfrac{\partial F_{y}}{\partial x}= & \dfrac{\partial\,}{\partial x}\left(\dfrac{x}{x^{2}+y^{2}}\right)\\
= & \dfrac{\left(x^{2}+y^{2}\right)-x\dfrac{\partial\,}{\partial y}\left(x^{2}+y^{2}\right)}{\left(x^{2}+y^{2}\right)^{2}}\\
= & \dfrac{y^{2}-x^{2}}{x^{2}+y^{2}}
\end{align*}
Agora vamos considerar um circuito fechado arbitraria, por simplicidade
escolhemos
\begin{align*}
\mathbf{r}(t)= & \cos t\,\mathbf{i}+\sin t\,\mathbf{j},\qquad0\leq t\leq2\pi\\
\mathbf{v}(t)= & -\sin t\,\mathbf{i}+\cos t\,\mathbf{j}
\end{align*}
a partir desta escolha podemos escrever
\[
\mathbf{F}(\mathbf{r})=-\sin t\,\mathbf{i}+\cos t\,\mathbf{j}
\]
de forma que
\begin{align*}
\oint_{c}\mathbf{F}\cdot d\mathbf{r}= & \int_{0}^{2\pi}\mathbf{F}(\mathbf{r})\cdot\mathbf{v}\, dt\\
= & \int_{0}^{2\pi}dt\\
= & 2\pi
\end{align*}
o que corrobora o fato de que mesmo que o rotacional seja nulo como
o domínio não é simplesmente conexo então o campo não será conservativo.

Em teoria está integral deve ser zero se for escolhido um caminho
tal que não contivesse o zero, desafortunadamente isto não pode ser
provado para um domínio arbitrário (tipo no I quadrante). A única
demonstração que temos é para um domínio definido por
\[
R=\left\{ \left(x,\, y\right)|x\in\mathbb{R},\, y\in\left(0,\,\infty\right)\right\} 
\]
O caminho de integração é o dado pela figura:

\begin{align*}
C_{1}:\,\mathbf{r}_{1}(t)= & t\,\mathbf{i},\qquad1\leq t\leq2\\
C_{2}:\,\mathbf{r}_{2}(t)= & 2\cos t\,\mathbf{i}+2\sin t\,\mathbf{j},\qquad0\leq t\leq\pi\\
C_{3}:\,\mathbf{r}_{3}(t)= & t\,\mathbf{i},\qquad-2\leq t\leq-1\\
C_{4}:\,\mathbf{r}_{4}(t)= & \cos t\,\mathbf{i}+\sin t\,\mathbf{j},\qquad\pi\leq t\leq0
\end{align*}
dessa forma temos

\[
\oint_{c}\mathbf{F}\cdot d\mathbf{r}=\int_{C_{1}}\mathbf{F}\cdot d\mathbf{r}+\int_{C_{2}}\mathbf{F}\cdot d\mathbf{r}+\int_{C_{3}}\mathbf{F}\cdot d\mathbf{r}+\int_{C_{4}}\mathbf{F}\cdot d\mathbf{r}
\]
assim, para o caminho $C_{1}$:
\[
\mathbf{F}=\left\langle -\dfrac{y}{x^{2}+y^{2}},\,\dfrac{x}{x^{2}+y^{2}}\right\rangle 
\]
\begin{align*}
\mathbf{v}_{1}= & \mathbf{i}\\
\mathbf{F}\left(\mathbf{r}_{1}\right)= & \dfrac{\mathbf{i}}{t}
\end{align*}
\begin{align*}
\int_{C_{1}}\mathbf{F}\cdot d\mathbf{r}= & \int_{1}^{2}\left(\dfrac{\mathbf{j}}{t}\right)\cdot\mathbf{i}\\
= & 0
\end{align*}
para o caminho $C_{3}$:
\begin{align*}
\mathbf{v}_{3}= & \mathbf{i}\\
\mathbf{F}\left(\mathbf{r}_{3}\right)= & \dfrac{\mathbf{i}}{t}
\end{align*}
\begin{align*}
\int_{C_{1}}\mathbf{F}\cdot d\mathbf{r}= & \int_{-2}^{-1}\left(\dfrac{\mathbf{j}}{t}\right)\cdot\mathbf{i}\\
= & 0
\end{align*}
para o caminho $C_{2}$:
\begin{align*}
\mathbf{v}_{2}= & -2\sin t\,\mathbf{i}+2\cos t\,\mathbf{j}\\
\mathbf{F}\left(\mathbf{r}_{2}\right)= & -\dfrac{1}{2}\sin t\,\mathbf{i}+\dfrac{1}{2}\cos t\,\mathbf{j}
\end{align*}
\begin{align*}
\int_{C_{1}}\mathbf{F}\cdot d\mathbf{r}= & \int_{0}^{\pi}\left(-\dfrac{1}{2}\sin t\,\mathbf{i}+\dfrac{1}{2}\cos t\,\mathbf{j}\right)\cdot\left(-2\sin t\,\mathbf{i}+2\cos t\,\mathbf{j}\right)dt\\
= & \int_{0}^{\pi}\left(\sin^{2}t+\cos^{2}t\right)dt\\
= & \pi
\end{align*}
para o caminho $C_{4}$:
\begin{align*}
\mathbf{v}_{4}= & -\sin t\,\mathbf{i}+\cos t\,\mathbf{j}\\
\mathbf{F}\left(\mathbf{r}_{4}\right)= & -\sin t\,\mathbf{i}+\cos t\,\mathbf{j}
\end{align*}
\begin{align*}
\int_{C_{1}}\mathbf{F}\cdot d\mathbf{r}= & \int_{\pi}^{0}\left(-\sin t\,\mathbf{i}+\cos t\,\mathbf{j}\right)\cdot\left(-\sin t\,\mathbf{i}+\cos t\,\mathbf{j}\right)dt\\
= & \int_{\pi}^{0}\left(\sin^{2}t+\cos^{2}t\right)dt\\
= & -\pi
\end{align*}
finalmente

\[
\oint_{c}\mathbf{F}\cdot d\mathbf{r}=0+\pi-\pi+0=0
\]
A função potencial associada é:
\begin{align*}
\dfrac{\partial f}{\partial x}= & -\dfrac{y}{y^{2}+x^{2}}\\
= & \int\left(-\dfrac{y}{y^{2}+x^{2}}\right)dx\\
= & -\arctan\dfrac{x}{y}+C\\
\dfrac{\partial f}{\partial y}= & \dfrac{\partial\,}{\partial y}\left(-\arctan\dfrac{x}{y}+C_{1}\right)\\
\dfrac{x}{y^{2}+x^{2}}= & -\dfrac{\dfrac{d\,}{dy}\left(\dfrac{x}{y}\right)}{1+\left(\dfrac{x}{y}\right)^{2}}+\dfrac{\partial C_{1}}{\partial y}\\
= & \dfrac{\left(\dfrac{x}{y^{2}}\right)}{1+\left(\dfrac{x}{y}\right)^{2}}+\dfrac{\partial C_{1}}{\partial y}\\
= & \dfrac{x}{y^{2}+x^{2}}+\dfrac{\partial C_{1}}{\partial y}\\
\dfrac{\partial C_{1}}{\partial y}= & Const.
\end{align*}
por tanto
\[
f=-\arctan\dfrac{x}{y}+C
\]



\subsection*{Exemplo 13}

Seja 
\[
\mathbf{F}(x,y)=\left\langle y,\, x\right\rangle 
\]
um campo vetorial conservativo, calcule a função potencial do qual
deriva utilizando o método da integral


\subsubsection*{SOL}

O método da integral consiste em admitir a independência da integral
de linha do caminho de integração, dessa forma se

\[
\mathbf{F}(x,y)=\vec{\nabla}f
\]
temos
\[
\int_{c}\mathbf{F}\cdot d\mathbf{r}=f(x,y)-f(A)
\]
vamos supor que o ponto inicial é 
\[
f(A)=f(0,0)=C
\]
de onde
\[
F(x,\, y)=\int_{c}\mathbf{F}\cdot d\mathbf{r}+C
\]
Para resolver essa integral vamos considerar uma reta que se estende
desde o ponto $A=(0,0)$ até o ponto $B=(x,y)$, assim nosso caminho
terá a seguinte equação
\begin{align*}
\mathbf{r}(t)= & xt\,\mathbf{i}+yt\,\mathbf{j},\qquad0\leq t\leq1\\
\mathbf{v}(t)= & x\,\mathbf{i}+y\,\mathbf{j}\\
\mathbf{F}(\mathbf{r})= & yt\,\mathbf{i}+xt\,\mathbf{j}\\
\mathbf{F}\cdot\mathbf{v}= & 2xyt
\end{align*}
de onde
\[
\int_{c}\mathbf{F}\cdot d\mathbf{r}=2\int_{0}^{1}xyt\, dt=xy
\]
assim
\[
F(x,\, y)=xy+C
\]

\end{document}
