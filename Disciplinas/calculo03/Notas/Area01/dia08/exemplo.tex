%% LyX 2.0.6 created this file.  For more info, see http://www.lyx.org/.
%% Do not edit unless you really know what you are doing.
\documentclass[12pt,brazil,a4papper]{article}
\usepackage[T1]{fontenc}
\usepackage[utf8x]{inputenc}
\usepackage{geometry}
\geometry{verbose,tmargin=2cm,bmargin=2cm,lmargin=2cm,rmargin=2cm}
\usepackage{amsmath}
\usepackage{amssymb}
\usepackage{mathdots}
\usepackage{setspace}
\usepackage{esint}
\onehalfspacing

\makeatletter
%%%%%%%%%%%%%%%%%%%%%%%%%%%%%% User specified LaTeX commands.

\usepackage{babel}
\usepackage{amsfonts}



 

\makeatother

\usepackage{babel}
\begin{document}

\subsection*{Exemplo 01}

Seja $f(x,y)=y-x$ e $C$ a curva definida pela função 
\[
\mathbf{r}(t)=\begin{cases}
\left(2t,\, t\right) & \text{se }0\leq t<1\\
\left(t+1,\,5-4t\right) & \text{se }1<t\leq3
\end{cases}
\]



\subsubsection*{SOL}

Como $\mathbf{r}(t)$ é uma função vetorial definida por partes de
classe $C^{1}$; onde cada parte é uma função de classe $C^{1}$,
temos
\[
\int_{\mathbf{r}}\, f\, ds=\int_{\mathbf{r}_{1}}\, f\, ds+\int_{\mathbf{r}_{2}}\, f\, ds
\]
onde $\mbox{\ensuremath{\mathbf{r}}}_{1}=\left(2t,\, t\right)$ com
$0\leq t\leq1$ e $\mbox{\ensuremath{\mathbf{r}}}_{2}=\left(t+1,\,5-4t\right)$
com $1\leq t\leq3$. Note que

\[
\mathbf{v}_{1}(t)=2\,\mathbf{i}+1\,\mathbf{j}
\]
de onde

\begin{align*}
\left|\mathbf{v}_{1}(t)\right|= & \sqrt{4+1}\\
= & \sqrt{5}
\end{align*}
Igualmente
\[
\mathbf{v}_{1}(t)=1\,\mathbf{i}-4\,\mathbf{j}
\]
de onde

\begin{align*}
\left|\mathbf{v}_{2}(t)\right|= & \sqrt{1+16}\\
= & \sqrt{17}
\end{align*}
de forma que
\begin{align*}
\int_{\mathbf{r}}\, f\, ds= & \int_{\mathbf{r}_{1}}\, f\, ds+\int_{\mathbf{r}_{2}}\, f\, ds\\
= & \int_{\mathbf{r}_{1}}\, f\,\left|\mathbf{v}_{1}(t)\right|\, dt+\int_{\mathbf{r}_{2}}\, f\,\left|\mathbf{v}_{2}(t)\right|\, dt\\
= & \int_{0}^{1}\left(y-x\right)\,\sqrt{5}\, dt+\int_{1}^{3}\left(y-x\right)\,\sqrt{17}\, dt\\
= & \int_{0}^{1}\left[\left(t\right)-\left(2t\right)\right]\,\sqrt{5}\, dt+\int_{1}^{3}\left[\left(5-4t\right)-\left(t+1\right)\right]\,\sqrt{17}\, dt\\
= & -\sqrt{5}\int_{0}^{1}t\, dt+\sqrt{17}\int_{1}^{3}\left[4-5t\right]\, dt\\
= & -\sqrt{5}\,\left.\dfrac{1}{2}t^{2}\right|_{0}^{1}+\sqrt{17}\left.4t-\dfrac{5}{2}t^{2}\right|_{1}^{3}\\
= & -\dfrac{\sqrt{5}}{2}+\sqrt{17}\left[8-20\right]\\
= & -\dfrac{\sqrt{5}}{2}-12\sqrt{17}
\end{align*}



\subsection*{Exemplo 02}

Calcule $\int_{C}{\displaystyle 4x\, dy+2y\, dz}$, onde $C$ consiste
do segmento de linha entre $(0,1,0)$ até $(0,1,1)$, seguido pelo
segmento $(0,1,1)$ até $(2,1,1)$ e seguido pela linha $(2,1,1)$
até $(2,4,1)$


\subsubsection*{SOL}

Parametrizando o segmento de linha entre $(0,1,0)$ até $(0,1,1)$:
\begin{align*}
\mathbf{r}_{a}= & \mathbf{r}_{0}+\mathbf{b}\, t\\
= & \mathbf{r}_{0}+\mathbf{\left(\mathbf{r}_{0}-\mathbf{r}_{1}\right)}\, t\\
= & (0,1,0)+\left[(0,1,1)-(0,1,0)\right]\, t\\
= & (0,1,0)+(0,0,1)\, t\\
= & \mathbf{j}+t\,\mathbf{k}
\end{align*}
de onde
\[
\mathbf{v}_{a}=\mathbf{k}
\]
para o caso do segmento que vai desde $(0,1,0)$ até $(2,1,1)$:

\begin{align*}
\mathbf{r}_{b}= & \mathbf{r}_{0}+\mathbf{b}\, t\\
= & \mathbf{r}_{0}+\mathbf{\left(\mathbf{r}_{0}-\mathbf{r}_{1}\right)}\, t\\
= & (0,1,1)+\left[(2,1,1)-(0,1,1)\right]\, t\\
= & (0,1,1)+(2,0,0)\, t\\
= & 2t\,\mathbf{i}+\mathbf{j}+\mathbf{k}
\end{align*}
de onde
\[
\mathbf{v}_{b}=2\mathbf{i}
\]
finalmente, para o caso do segmento que vai desde $(2,1,1)$ até $(2,4,1)$:

\begin{align*}
\mathbf{r}_{c}= & \mathbf{r}_{0}+\mathbf{b}\, t\\
= & \mathbf{r}_{0}+\mathbf{\left(\mathbf{r}_{0}-\mathbf{r}_{1}\right)}\, t\\
= & (2,1,1)+\left[(2,4,1)-(2,1,1)\right]\, t\\
= & (2,1,1)+(0,3,0)\, t\\
= & 2\,\mathbf{i}+3t\,\mathbf{j}+\mathbf{k}
\end{align*}
de onde
\[
\mathbf{v}_{c}=3\mathbf{j}
\]
Para resolvermos as integra $\int_{C}4x\, dy+2y\, dz$ devemos observar
que

\begin{align*}
\int_{C}4x\, dy+2y\, dz= & \int_{\mbox{\ensuremath{\mathbf{r}}}_{a}}4x\, dy+2y\, dz+\int_{\mbox{\ensuremath{\mathbf{r}}}_{b}}4x\, dy+2y\, dz+\\
 & \int_{\mathbf{r}_{c}}4x\, dy+2y\, dz
\end{align*}
Para o caso de qualquer uma das curvas temos
\begin{align*}
ds= & \left|\mathbf{v}\right|dt\\
= & \dfrac{dx}{dt}\, dt+\dfrac{dy}{dt}\, dt+\dfrac{dz}{dt}\, dt\\
= & dx+dy+dz
\end{align*}
de forma que para o caso da trajetória $\mathbf{r}_{a}$ temos
\begin{align*}
dx= & 0\, dt\\
dy= & 0\, dt\\
dz= & 1\, dt
\end{align*}
para o caso da trajetória $\mathbf{r}_{b}$ temos
\begin{align*}
dx= & 2\, dt\\
dy= & 0\, dt\\
dz= & 0\, dt
\end{align*}
e para o caso da trajetória $\mathbf{r}_{c}$ temos
\begin{align*}
dx= & 0\, dt\\
dy= & 3\, dt\\
dz= & 0\, dt
\end{align*}
assim então
\begin{align*}
\int_{C}4x\, dy+2y\, dz= & \int_{\mathbf{r}_{a}}4x\, dy+2y\, dz+\int_{\mathbf{r}_{b}}4x\, dy+2y\, dz\\
 & \int_{\mathbf{r}_{c}}4x\, dy+2y\, dz\\
= & \int_{\mathbf{r}_{a}}4\left(0\right)\left(0\right)\, dt+2(1)\, dt+\int_{\mathbf{r}_{b}}4\left(2\right)(0)\, dt+2(1)(0)\, dt\\
 & \int_{\mathbf{r}_{c}}4\left(2\right)(3)\, dt+2(1)(0)\, dt\\
= & 2\int_{\mathbf{r}_{a}}dt+24\int_{\mathbf{r}_{c}}\, dt
\end{align*}
A parametrização utilizada para o I e III segmento implica que o percurso
é percorrido para $0\leq t\leq1$, dessa forma as integrais ficam
\begin{align*}
\int_{C}4x\, dy+2y\, dz= & 2\int_{0}^{1}dt+24\int_{0}^{1}\, dt\\
= & 26
\end{align*}


\[
\mathbf{v}_{1}=\,\mathbf{i}+\,\mathbf{j}
\]



\subsection*{Exemplo 3}

Calcule ${\displaystyle \int_{c}\left(x+y+z\right)ds}$, onde $C$
é a hélice re raio 1 definido entre $0\leq t\leq\pi$


\subsection*{SOL}

Em se tratando de uma hélice a parametrização é
\[
\mathbf{r}(t)=\cos\, t\,\mathbf{i}+\sin\, t\,\mathbf{j}+t\,\mathbf{k}
\]
de onde
\begin{align*}
\mathbf{r}(t)= & -\sin\, t\,\mathbf{i}+\cos\, t\,\mathbf{j}+\mathbf{k}\\
\left|\mathbf{v}(t)\right|= & \sqrt{\sin^{2}t+\cos^{2}t+1}\\
= & \sqrt{2}
\end{align*}
assim
\begin{align*}
\int_{c}\left(x+y+z\right)ds= & \sqrt{2}\int_{0}^{\pi}\left(\cos\, t+\sin\, t+t\right)\, dt\\
= & \sqrt{2}\left[\sin\, t-\cos\, t+\dfrac{1}{2}t^{2}\right]_{0}^{\pi}\\
= & \sqrt{2}\left[\left(1-0+\dfrac{\pi^{2}}{2}\right)-\left(0-1\right)\right]\\
= & \sqrt{2}\left(2+\dfrac{\pi^{2}}{2}\right)
\end{align*}



\subsection*{Exemplo 04}

Calcular $\int_{C}xy\, ds$, onde $C$ é a interseção das superfícies
$x^{2}+y^{2}=4$ e o plano $y+z=8$


\subsubsection*{SOL}

A equação paramétrica da interseção se obtém isolando o $z$
\[
z=8-y
\]
e utilizando a parametrização da circunferência
\[
\begin{cases}
x= & 2\cos t\\
y= & 2\sin t\\
z= & 8-2\sin t
\end{cases}\;\;\;0\leq t\leq2\pi
\]
de onde
\begin{align*}
\mathbf{v}(t)= & -2\sin\, t\,\mathbf{i}+2\cos\, t\,\mathbf{j}-2\cos\mathbf{k}\\
\left|\mathbf{v}(t)\right|= & 2\sqrt{\sin^{2}t+\cos^{2}+\cos^{2}t}\\
= & 2\sqrt{1+\cos^{2}t}\\
= & \cos\dfrac{t}{2}
\end{align*}
assim
\[
\int_{C}xy\, ds=\int_{0}^{2\pi}2\cos t\,2\sin t\,2\sqrt{1+\cos^{2}t}\, dt
\]
utilizando 
\begin{align*}
u= & 1+\cos^{2}t\\
du= & 2\cos t\,\sin t\, dt
\end{align*}
de onde
\begin{align*}
\int2\cos t\,2\sin t\,2\sqrt{1+\cos^{2}t}\, dt= & 4\int u^{1/2}dt\\
= & \dfrac{2}{3u^{3/2}}+C\\
= & \dfrac{2}{3\left(1+\cos^{2}t\right)^{3/2}}+C
\end{align*}
de forma que
\begin{align*}
\int_{C}xy\, ds= & \left.\dfrac{2}{3\left(1+\cos^{2}t\right)^{3/2}}\right|_{0}^{2\pi}\\
= & 0
\end{align*}



\subsection*{Exemplo 05}

Avalie a integral $\int_{C}x^{2}y\, ds$, onde $C$ está dada pela
parametrização $\mathbf{r}(t)=3\cos\, t\,\mathbf{i}+3\sin\, t\,\mathbf{j}$,
$0\leq t\leq\pi/2$. Seguidamente mostre que se utilizamos a parametrização
$\mathbf{r}(t)=\sqrt{9-y^{2}}\,\mathbf{i}+y\,\mathbf{j}$, $0\leq y\leq3$,
obteremos o mesmo resultado.


\subsubsection*{SOL}

\begin{align*}
\mathbf{r}(t)= & 3\cos\, t\,\mathbf{i}+3\sin\, t\,\mathbf{j}\\
\mathbf{v}(t)= & 3\sin\, t\,\mathbf{i}+3\cos\, t\,\mathbf{j}\\
\left|\mathbf{v}(t)\right|= & 3
\end{align*}
dessa forma
\begin{align*}
\int_{C}x^{2}y\, ds= & 81\int_{0}^{\frac{\pi}{2}}\cos^{2}t\,\sin t\, dt\\
= & -81\int_{1}^{0}u^{2}du\\
= & \dfrac{81}{3}\,\left.u^{3}\right|_{0}^{1}\\
= & 27
\end{align*}
Utilizando a outra parametrização
\begin{align*}
\mathbf{r}(y)= & \sqrt{9-y^{2}}\,\mathbf{i}+y\,\mathbf{j}\\
\mathbf{v}(y)= & -\dfrac{y}{\sqrt{9-y^{2}}}\,\mathbf{i}+\,\mathbf{j}\\
\left|\mathbf{v}(y)\right|= & \dfrac{3}{\sqrt{9-y^{2}}}\\
ds= & \dfrac{3}{\sqrt{9-y^{2}}}dy
\end{align*}
assim
\begin{align*}
\int_{C}x^{2}y\, ds= & 3\int_{0}^{3}\dfrac{\left(9-y^{2}\right)y}{\sqrt{9-y^{2}}}\, dy\\
= & 3\int_{0}^{3}y\sqrt{9-y^{2}}\, dy\\
= & -\dfrac{3}{2}\int_{3}^{0}u^{1/2}du\\
= & \left.u^{3/2}\right|_{0}^{9}\\
= & 27
\end{align*}



\subsection*{Exemplo 06}

Calcule a área da superfície estendida para cima do círculo $x^{2}+y^{2}=1$
no plano $xy$ até a parábola cilíndrica $z=1-x^{2}$


\subsubsection*{SOL}

A área da superfície é
\[
A=\int_{C}\left(1-x^{2}\right)ds
\]
onde $C$ é a circunferência, onde
\[
\mathbf{r}(t)=\cos\, t\,\mathbf{i}+\sin\, t\,\mathbf{j}+t\,\mathbf{k}
\]
de onde
\begin{align*}
\mathbf{r}(t)= & -\sin\, t\,\mathbf{i}+\cos\, t\,\mathbf{j}\\
\left|\mathbf{v}(t)\right|= & \sqrt{\sin^{2}t+\cos^{2}t}\\
= & 1
\end{align*}
assim
\begin{align*}
A= & \int_{0}^{2\pi}\left(1-\cos^{2}t\right)dt\\
= & \int_{0}^{2\pi}\sin^{2}t\, dt
\end{align*}
utilizando a identidade $2\sin^{2}t=1-\cos2t$, assim
\begin{align*}
A= & \dfrac{1}{2}\int_{0}^{2\pi}\left(1-\cos2t\right)dt\\
= & \pi
\end{align*}

\end{document}
