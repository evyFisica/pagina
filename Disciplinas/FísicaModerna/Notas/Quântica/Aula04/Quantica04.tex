\documentclass[12pt,brazil,table]{beamer}
\usepackage[utf8]{inputenc}
\usepackage[portuguese]{babel}
\setcounter{secnumdepth}{3}
\setcounter{tocdepth}{3}
\usepackage{movie15}
\usepackage{graphicx}
\setbeamercovered{transparent}
\usetheme{Boadilla}
\usepackage{times}
\usefonttheme{structurebold}
\usepackage{pgf}
\beamertemplatetransparentcovereddynamic
%\usepackage{multimedia}
\usepackage{animate}

\usepackage{unicode-math}
\usepackage{mathrsfs}

\usepackage{braket}


\usepackage{cancel}
\usepackage{multicol}
\usepackage{hyperref}

\usepackage{tabularray}
\definecolor{cabezeraTabela}{RGB}{230, 229, 192}
\definecolor{linhaPar}{RGB}{245, 245, 245}
\definecolor{linhaImpar}{RGB}{253, 252, 212}
\definecolor{linhaTabela}{RGB}{255, 106, 0}

% \usepackage{cellspace,tabularx}
% \setlength\cellspacetoplimit{5pt}
% \setlength\cellspacebottomlimit{5pt}
% \newcolumntype{C}{>{\centering\arraybackslash}X}
% \addparagraphcolumntypes{C}
% \usepackage[many]{tcolorbox}
% \newtcolorbox{tctabularx}[1]{%
%           enhanced,
%           fonttitle=\sffamily\bfseries, fontupper=\small\sffamily,
%           colback=blue!10, colframe=blue,
%           #1,
%           before upper app={\rowcolor{blue!60}},
%                             }% end tctabularx


\title{Revisão de Mecânica Quântica}
\subtitle{Aula do MNPEF - Polo 41}
\author{Evy A. Salcedo Torres}
\date{\today}

\begin{document}

%%%%%%%%%%%%%%%%  SLIDE 1 %%%%%%%%%%%%%%%%%%%%%%%%%%%

\begin{frame}
  \titlepage
\end{frame}

%%%%%%%%%%%%%%%%  SLIDE 2 %%%%%%%%%%%%%%%%%%%%%%%%%%%


\begin{frame}
  \frametitle{Outra notação}
  \fontsize{7pt}{11pt}\selectfont
  
  As funções de onda são vetores do chamado espaço de Hilbert $\mathscr{L}$. Os vetores desse espaço vetorial são funções complexas definidas no intervalo $[a,\, b]$ as quais são funções de quadrado integrável, isto é, a integral que representa o produto interno 
  \[
   \left|\,\psi \, \right|^2 = \int_a^b \left| \, \Psi(x)  \, \right|^2\,dx = \int_a^b \psi^*(x) \psi(x)\, dx,
  \]
  resulta em um valor finito.\\  
  Uma representação desses vetores foi introduzida por P. M. Dirac,
  \[
   \ket{\psi_i} \in \mathscr{L}
  \]
  onde
  \[
   \left(\ket{\psi_i}\right)^*=\bra{\psi_i}
  \]
  onde o produto internos é escrito como
  \[
   \braket{\psi_i|\psi_j} = \braket{\psi_j|\psi_i}
  \]
  Nesta notação fica claro que um vetor admite uma fase arbitraria
  \[
    \left.\begin{array}{rcl}
      \ket{\psi_j} & \longrightarrow & e^{i\delta}\ket{\psi_j}\\
      \bra{\psi_k} & \longrightarrow & \bra{\psi_k}e^{-i\delta}
    \end{array}\right\} \quad \braket{\psi_k|e^{-i\delta}e^{i\delta}|\psi_j}=\braket{\psi_k|\psi_j}
  \]
  Podemos afirmar que o estado de uma partícula é representado por um ${\psi}$ e a amplitude de probabilidade, para um partícula no estado $\ket{\psi}$ ser encontrada no estado $\bra{\phi}$ é dada por
  \[
   P = \braket{\psi|\phi}
  \]
     
\end{frame} 

%%%%%%%%%%%%%%%%  SLIDE 3 %%%%%%%%%%%%%%%%%%%%%%%%%%%


\begin{frame}
  \frametitle{Outra notação}
  \fontsize{7pt}{11pt}\selectfont
  
  \begin{columns}[T]
   \column{0.49\linewidth}


  Suponha que $\hat{A}$ seja o operador que representa o observável $A$ que ao ser realizada uma medida $a_1$, $a_2$, $a_3$, $\ldots$.  A representação quântica geral de um estado é expressado na forma do vetor $\ket{\psi}$, pode ser escrito como uma superposição $\ket{a_1}$, $\ket{a_2}$, $\ket{a_3}$, $\ldots$ que resulta de uma medida, isto é \vspace*{-0.25cm}
  \[
  \begin{align*}
    \ket{\psi} &= c_1\ket{a_1} + c_2\ket{a_2} + c_3\ket{a_3} + \ldots\\
    &= \sum_n c_n\ket{a_n}
  \end{align*}\vspace*{-0.5cm}
  \]
  igualmente
  \[
    \bra{\psi} = \sum_n c_n^* \bra{a_n}
  \]
  note que\vspace*{-0.5cm}
  \[
   c_n=\braket{a_n|\psi}=\int_{-\infty}^\infty a^*(x)\psi(x)\, dx\vspace*{-0.25cm}
  \]
  que corresponde probabilidade de se obter $a_n$ se é realizada uma medida de $A$ estando a partícula/sistema no estado $\bra{\psi}$\\
  
  Esperamos que\vspace*{-0.5cm}
  \[
    \begin{align*}
      \braket{a_i|a_j} &= \begin{cases}
      0 & \text{se}\quad i\neq j\\
      1 & \text{se}\quad i = j
      \end{cases}\\
      &= \delta_{ij}
    \end{align*}\vspace*{-0.5cm}
  \]
   \column{0.49\linewidth}\vspace*{-0.5cm}
  Note que  \vspace*{-0.5cm}
  \[
  \begin{align*}
    \braket{a_i|\psi} &= \bra{a_i}\left( \sum_j c_j \ket{a_j}  \right)\\
    &= \sum_j c_j\braket{a_i|a_j}\\
    &= \sum_j c_j\delta_{ij}\\
    &=c_i
  \end{align*}\vspace*{-0.5cm}
  \]
  de forma que\vspace*{-0.5cm}
  \[
  \begin{align*}
    \ket{\psi} &=  \sum_j c_j \ket{a_j}\\
    &= \sum_j \left(\braket{a_j|\psi}\right)\ket{a_j}\\
    &= \sum_j \ket{a_j}\braket{a_j|\psi}
  \end{align*}\vspace*{-0.5cm}
  \]
  Como $\braket{\psi|\psi}=1$\vspace*{-0.25cm}
  \[
  \begin{align*}
    1 &= \left( \sum_i c_i^* \bra{a_i}  \right)\left( \sum_j c_j \ket{a_j}  \right)\\
     &= \sum_i\sum_j c_i^*c_j \delta_{ij}\\
    &= \sum_i \left|\, c_i  \,\right|^2
  \end{align*}\vspace*{-0.5cm}
  \]
  
  \end{columns}


\end{frame} 

%%%%%%%%%%%%%%%%  SLIDE 4 %%%%%%%%%%%%%%%%%%%%%%%%%%%

\begin{frame}
  \frametitle{Outra notação}
  \fontsize{7pt}{11pt}\selectfont
  
  \begin{columns}[T]
  \column{0.49\linewidth}
  
  Se é realizada uma medida de $\hat{A}$, o valore médio de um observável $A$ para uma partícula no estado $\bra{\psi}$ é dada por\vspace*{-0.5cm}
  \[
    \begin{align*}
      \Delta A &= \sqrt{\langle \left( A-  \langle A\rangle\right)^2 \rangle}\\
      &=\sqrt{\langle A^2 \rangle - 2\langle A\rangle\langle A\rangle + \langle A\rangle ^2}\\
      &= \sqrt{\langle A^2 \rangle - A\rangle ^2}
    \end{align*}\vspace*{-0.25cm}
  \]
  onde $\langle A^2 \rangle = \sum_n \left|\, c_i  \,\right|^2 a_i^2$\\
  
  No caso de trabalharmos com variáveis contínuas, como a posição, as somas devem ser expressadas como integrais, por exemplo\vspace*{-0.25cm}
  \[
   \bra{\psi} = \int_{-\infty}^\infty dx \ket{x}\braket{x|\psi}\vspace*{-0.25cm}
  \]
  contudo, é verdade que se satisfaz\vspace*{-0.25cm}
  \[
    \hat{x}\ket{x}=x\ket{x}\vspace*{-0.35cm}
  \]
  A translação da posição de uma partícula é dada por\vspace*{-0.35cm}
  \[
   \hat{T}(a)\ket{x}=\ket{x+a}\vspace*{-0.25cm}
  \]
  Uma translação finita resulta da aplicação de um número infinito de translações infinitesimais
  
  \column{0.49\linewidth}\vspace*{-1cm}
  \[
    \begin{align*}
      \hat{T}(a) &= \lim_{N\to \infty} \left[ 1- \left(\dfrac{i}{\hbar}\right)\hat{p}_x\left(\dfrac{a}{N}\right)\right]^N\\
      &= \exp\left( -\dfrac{i\hat{p}_x}{\hbar}a \right)\\
      &= \left[ 1 - \dfrac{i\hat{p}_x a}{\hbar} + \dfrac{1}{2!}\left( \dfrac{i\hat{p}_x a}{\hbar} \right)^2 + \cdots \right]
    \end{align*}\vspace*{-0.25cm}    
  \]
  No caso de um deslocamento infinitesimal $\delta x$\vspace*{-0.25cm}
  \[
    \begin{align*}
    \hat{T}(\delta x)\ket{\psi} &= \int dx \ket{x+\delta x}\braket{x|\psi}\\
    &=\int dx' \ket{x'}\braket{x'-\delta x'|\psi}
    \end{align*}\vspace*{-0.25cm}
  \]
  onde foi feita a mudança de variáveis $x'=x+\delta x$. Como $\braket{x'-\delta x'|\psi}=\psi\left(x'-\delta x'\right)$, expandindo em série de Taylor ao redor de $x'$ até primeira ordem\vspace*{-0.25cm}
  \[
    \begin{align*}
      \psi\left(x'-\delta x'\right) &= \psi \left(x'\right) - \delta x \dfrac{\partial\;}{\partial x'}\psi \left(x'\right)\\
      &= \braket{x'|\psi} - \delta x \dfrac{\partial\;}{\partial x'}\braket{x'|\psi} 
    \end{align*}\vspace*{-0.5cm}
  \]
  substituindo\vspace*{-0.25cm}
  \[
    \hat{T}(\delta x)\ket{\psi} = \int dx' \ket{x'}\left( \braket{x'|\psi} - \delta x \dfrac{\partial\;}{\partial x'}\braket{x'|\psi} \right)\vspace*{-0.25cm}
  \]
  como, da definição de translação finita sabemos que ($N=1$)

  \end{columns}

\end{frame} 


%%%%%%%%%%%%%%%%  SLIDE 5 %%%%%%%%%%%%%%%%%%%%%%%%%%%

\begin{frame}
  \frametitle{Outra notação}
  \fontsize{7pt}{11pt}\selectfont
  
  \begin{columns}[T]
  \column{0.49\linewidth}
  \[
    \hat{T}(\delta x)\ket{\psi} = \left( 1- \dfrac{i}{\hbar}\hat{p}_x\delta x\right)\ket{\psi}
  \]
  assim, por comparação\vspace*{-0.25cm}
  \[
    \begin{align*}
    \dfrac{i}{\hbar}\hat{p}_x\ket{\psi} &= \int dx'\ket{x'}\dfrac{\partial\;}{\partial x'}\braket{x'|\psi}\\
    \braket{x|p_x|\psi} &= \dfrac{\hbar}{i} \int dx'\braket{x|x'}\dfrac{\partial\;}{\partial x'}\braket{x'|\psi}\\
    \braket{x|p_x|\psi} &= \dfrac{\hbar}{i} \int \delta (x-x')\dfrac{\partial\;}{\partial x'}\braket{x'|\psi}\\
    \braket{x|p_x|\psi} &= \dfrac{\hbar}{i} \dfrac{\partial\;}{\partial x}\braket{x|\psi}
    \end{align*}\vspace*{-0.25cm}
  \]
  se escolhemos o estado $\bra{\psi}=\bra{x'}$, então\vspace*{-0.25cm}
  \[
   \braket{x|p_x|x'} = \dfrac{\hbar}{i} \dfrac{\partial\;}{\partial x}\braket{x|x'} = \dfrac{\hbar}{i} \dfrac{\partial\;}{\partial x}\delta (x-x')\vspace*{-0.25cm}
  \]
  ou, calculamos o produto interno\vspace*{-0.25cm}
  \[
    \begin{align*}
      \langle p_x \rangle &= \braket{\psi|p_x|\psi}\\
      &= \dfrac{\hbar}{i} \int dx'\braket{\psi|x'}\dfrac{\partial\;}{\partial x'}\braket{x'|\psi}\\
      &= \dfrac{\hbar}{i} \int dx' \psi(x')\dfrac{\partial\;}{\partial x'}\psi(x')
    \end{align*}
  \]




  \column{0.51\linewidth}\vspace*{-1cm}
  \[
   \langle p_x \rangle = \dfrac{\hbar}{i} \int dx \psi(x)\dfrac{\partial\;}{\partial x}\psi(x)\vspace*{-0.05cm}
  \]
  onde mudamos de variável.  De todo o anterior podemos inferir que o operador momento é dado por\vspace*{-0.25cm}
  \[
   \hat{p} \xrightarrow[\text{na base }x]{} \dfrac{\hbar}{i}\dfrac{\partial\;}{\partial x}\vspace*{-0.25cm}
  \]
  Além da base das posições, temos a base dos momentos, nesse caso\vspace*{-0.25cm}
  \[
    \ket{\psi} = \int dp\ket{p}\braket{p|\psi}\vspace*{-0.25cm}
  \]
  e, igualmente é válido que\vspace*{-0.25cm}
  \[
   \hat{p}_x\ket{p} = p\ket{p}\vspace*{-0.25cm}
  \]
  utilizando o operador momento, na base das posições  \vspace*{-0.25cm}
  \[
    \begin{align*}
    \braket{x|p_x|p} &= \dfrac{\hbar}{i} \dfrac{\partial\;}{\partial x}\braket{x|p}\\
    p\braket{x|p} &=\dfrac{\hbar}{i} \dfrac{\partial\;}{\partial x}\braket{x|p}\\
    \dfrac{d\braket{x|p}}{\braket{x|p}} &= \dfrac{ip}{\hbar}dx\\
    \braket{x|p} &= \dfrac{1}{\sqrt{2\pi\hbar}}\exp\left(\dfrac{i\,p x}{\hbar}\right)
    \end{align*}\vspace*{-0.15cm}
  \]
  expressão já normalizada

  \end{columns}

\end{frame} 
  

%%%%%%%%%%%%%%%%  SLIDE 6 %%%%%%%%%%%%%%%%%%%%%%%%%%%

\begin{frame}
  \frametitle{Outra notação}
  \fontsize{7pt}{11pt}\selectfont
  
  \begin{columns}[T]
  

  \column{0.5\linewidth}
  O resultado final\vspace*{-0.25cm}
  \[
    \begin{align*}
      \braket{p|\psi} &= \int dx \braket{p|x}\braket{x|\psi}\\
      &= \int dx \dfrac{1}{\sqrt{2\pi\hbar}}\exp\left(\dfrac{i\,p x}{\hbar}\right)\braket{x|\psi}\\
      \braket{x|\psi} &= \int dp \braket{x|p}\braket{p|\psi}\\
      &= \int dp \dfrac{1}{\sqrt{2\pi\hbar}}\exp\left(\dfrac{i\,p x}{\hbar}\right)\braket{p|\psi}
    \end{align*}    
  \]
  no qual mostra que $\braket{p|\psi}$ e $\braket{x|\psi}$ correspondem a uma par de transformadas de Fourier.
  

  Similarmente podemos definir a evolução de um estado é descrito por\vspace*{-0.3cm}
  \[
   \hat{U}(t)\ket{\psi(0)} = \ket{\psi(t)}\vspace*{-0.3cm}
  \]
  sendo $\hat{U}(t)$ o operador de evolução temporal.\vspace*{-0.3cm}
  \[
   \hat{U}(dt) = 1 - \dfrac{\hat{H}}{\hbar}dt\vspace*{-0.25cm}
  \]
  Pode ser demostrado que\vspace*{-0.25cm}
  \[
    \hat{U}(t) = \exp\left( -\dfrac{\hat{H}}{\hbar}t \right)
  \]
  sendo o operador $\hat{H}$ é o gerador de translação temporal
  
  
  \column{0.5\linewidth}\vspace*{-0.8cm}
  \[
      \hat{U}(t) = \left[ 1 - \dfrac{\hat{H}t}{\hbar} + \dfrac{1}{2!}\left( \dfrac{\hat{H}t}{\hbar} \right)^2 + \cdots \right]
  \]
  sendo o operador $\hat{H}$ é o gerador de translação temporal. De fato, verifica-se que $\hat{H}$ é o operador energia (independente do tempo), pelo que é de se esperar\vspace*{-0.15cm}
  \[
    \hat{H}\ket{E} = E\ket{E}\vspace*{-0.15cm}
  \]
  se supomos que $\ket{\psi (0)}=\ket{E}$, então\vspace*{-0.25cm}
  \[
    \begin{align*}
   \ket{\psi(t)} &= \exp\left( -\dfrac{\hat{H}}{\hbar}t \right)\ket{\psi (0)}\\
    &= \exp\left( -\dfrac{\hat{H}}{\hbar}t \right)\ket{E}\\
    &= \exp\left( -\dfrac{E}{\hbar}t \right)\ket{E}
    \end{align*}\vspace*{-0.25cm}
  \]
  O operador Hamiltoniano é definido como\vspace*{-0.3cm}
  \[
    \begin{align*}
    \hat{H}\ket{\psi(t)} &= i\hbar \dfrac{d\,}{dt}\ket{\psi(t)}\\
    \braket{x|\hat{H}|\psi(t)} &= \bra{x}\left[\dfrac{\hat{p}_x^2}{2m} + V\left(\hat{x}\right) \right]\ket{\psi(t)}\\
    &= \left[-\dfrac{\hbar^2}{2m}\dfrac{\partial^2 }{\partial  x^2} + V\left(x\right)\right]\braket{x|\psi(t)}
    \end{align*}
  \]
  
  \end{columns}

\end{frame}
  

%%%%%%%%%%%%%%%%  SLIDE 7 %%%%%%%%%%%%%%%%%%%%%%%%%%%

\begin{frame}
  \frametitle{Outra notação}
  \fontsize{7pt}{11pt}\selectfont
  
  \begin{columns}[T]
  \column{0.5\linewidth}
  
  onde usamos $\bra{x}V\left(\hat{x}\right) = \bra{x}V\left(x\right)$ e como $\psi (x,t)=\braket{x|\psi(t)}$, então se obtêm\vspace*{-0.25cm}
  \[
    \left[-\dfrac{\hbar^2}{2m}\dfrac{\partial^2 }{\partial  x^2} + V\left(x\right)\right]\psi (x,t)=i\hbar \dfrac{d\,}{dt}\psi (x,t)\vspace*{-0.25cm}
  \]
  se escolhemos um estado $\psi(t)$ como sendo um autoestado de energia, isto é $\ket{\psi(t)}=\ket{E}e^{-iEt/\hbar}$, então\vspace*{-0.25cm}
  \[
    \psi_E (x,t) =\braket{x|E}e^{-iEt/\hbar}\vspace*{-0.25cm}
  \]
  de forma que a equação de Schrödinger assume a forma
  \[
    \begin{align*}
      \left[-\dfrac{\hbar^2}{2m}\dfrac{\partial^2 }{\partial  x^2} + V\left(x\right)\right]\braket{x|E}e^{-iEt/\hbar} \qquad\quad&\\
    \qquad = E\braket{x|E}e^{-iEt/\hbar} 
    \end{align*}\vspace*{-0.25cm}
  \]
  Esta equação também resulta da projeção a equação de autovalor de energia\vspace*{-0.25cm}
  \[
   \hat{H}\ket{E}=E\ket{E}\vspace*{-0.25cm}
  \]
  dentro do espaço das posições\vspace*{-0.25cm}
  \[
   \braket{x|\hat{H}|E}=E\braket{x|E}
  \]
  
  \column{0.5\linewidth}\vspace*{-0.8cm}
  como $\psi_E (x)=\braket{x|E}$ e removendo o subíndice
  \[
   \left[-\dfrac{\hbar^2}{2m}\dfrac{\partial^2 }{\partial  x^2} + V\left(x\right)\right]\psi (x) = E\psi (x)
  \]
  \textbf{Oscilador Harmônico}\\\vspace*{-0.25cm}
  
  \[
    \hat{H} = \dfrac{\hat{p}_x}{2m}+\dfrac{1}{2}m\omega^2 \hat{x}\vspace*{-0.2cm}
  \]
  mantendo que $\left[ \hat{x},\, \hat{H}\right]=i\hbar$. Usando\vspace*{-0.25cm}
  \[
    \begin{align*}
      \hat{a} &= \sqrt{\dfrac{m\omega}{2\hbar}}\left(\hat{x} + \dfrac{i}{m\omega}\hat{p}_x\right)\\
      \hat{a}^\dagger &= \sqrt{\dfrac{m\omega}{2\hbar}}\left(\hat{x} - \dfrac{i}{m\omega}\hat{p}_x\right)
    \end{align*}\vspace*{-0.25cm}
  \]
  que são os operadores de criação e destruição, os quais verificam $\left[ \hat{a},\, \hat{a}^\dagger \right]=1$. Invertendo essa definição, obtemos\vspace*{-0.35cm}
  \[
    \begin{align*}
      \hat{x} &= \sqrt{\dfrac{\hbar}{2 m\omega}}\left(\hat{a} + \hat{a}^\dagger\right)\\
      \hat{p}_x &= -i\sqrt{\dfrac{ m\omega\hbar}{2}}\left(\hat{a} - \hat{a}^\dagger\right)
    \end{align*}\vspace*{-0.25cm}
  \]
  A partir disso\vspace*{-0.5cm}
  \[
   \hat{H} = \hbar\omega \left(\hat{a}\hat{a}^\dagger + \dfrac{1}{2} \right)
  \]
  \end{columns}

\end{frame}

%%%%%%%%%%%%%%%%  SLIDE 8 %%%%%%%%%%%%%%%%%%%%%%%%%%%

\begin{frame}
  \frametitle{Outra notação}
  \fontsize{7pt}{11pt}\selectfont
  
  \begin{columns}[T]
  \column{0.47\linewidth}
  defini-se o operador número, $\hat{N} = \hat{a} + \hat{a}^\dagger$, o qual tem um auto vetor $\ket{n}$ e verifica\vspace*{-0.25cm}
  \[
   \hat{N}\ket{n} = n\ket{n},\quad \left[ \hat{N},\, \hat{a}\right] = -\hat{a},\quad \left[ \hat{N},\, \hat{a}^\dagger\right] = \hat{a}^\dagger\vspace*{-0.25cm}
  \]
  pode-se verificar que\vspace*{-0.35cm}
  \[
    \begin{align*}
      \hat{N}\left(\hat{a}^\dagger \ket{n}\right) &= \left( n+1 \right)\left(\hat{a}^\dagger \ket{n}\right)\\
     \Rightarrow \hat{a}^\dagger \ket{n} &=\sqrt{n+1}\,\ket{n+1}\vspace*{-0.25cm}
    \end{align*}   
  \]\vspace*{-0.15cm}
  \[
    \begin{align*}
      \hat{N}\left(\hat{a} \ket{n}\right) &= \left( n-1 \right)\left(\hat{a} \ket{n}\right)\\
     \Rightarrow \hat{a} \ket{n} &=\sqrt{n}\,\ket{n-1}\vspace*{-0.25cm}
    \end{align*}   
  \]
  (os autovalores tem esses valores devidos à normalização) o que vai resultar em que\vspace*{-0.25cm}
  \[
    \ket{n} = \dfrac{\left( \hat{a}^\dagger \right)^n}{\sqrt{n!}}\,\ket{0} \vspace*{-0.25cm}
  \]

  substituindo no Hamiltoniano e aplicando sobre os vetores de $\ket{n}$
  
  \column{0.53\linewidth}\vspace*{-0.8cm}
  \[
    \begin{align*}
      \hat{H}\ket{n} &= \hbar\omega \left(\hat{N} + \dfrac{1}{2} \right)\ket{n}\\
      &= \hbar\omega \left (n + \dfrac{1}{2} \right)\ket{n}\\
      &= E_n\ket{n}\qquad\qquad\qquad n=0,1,2,\cdots
    \end{align*}\vspace*{-0.25cm}
  \]
  
  \textbf{Simetria de traslação e rotação}\\ \vspace*{-0.75cm}
  \[
    \begin{array}{ccc}
    \ket{\mathbf{r}} = \ket{x,y,z}&&\\
    \hat{x}\ket{\mathbf{r}} = x\ket{\mathbf{r}},& \hat{y}\ket{\mathbf{r}} = y\ket{\mathbf{r}},& \hat{z}\ket{\mathbf{r}} = z\ket{\mathbf{r}}\vspace*{-0.25cm}
    \end{array}    
  \]
  \[
    \begin{align*}
      \ket{\psi} &= \iiint dxdydz \ket{x,y,z}\braket{x,y,z|\psi}\\
      &= \iiint d^3r \ket{\mathbf{r}}\braket{\mathbf{r}|\psi}
    \end{align*}\vspace*{-0.25cm}
  \]
  onde\vspace*{-0.25cm}
  \[
    \left[\hat{p}_x,\, \hat{p}_y\right]=0,\qquad \left[\hat{x}_i,\,\hat{p}_j\right]=i\hbar\delta_{ij}\vspace*{-0.25cm}
  \]
  \[
    \hat{T}(\mathbf{a}) = e^{-i\hat{\mathbf{p}}\cdot \hat{\mathbf{a}}/\hbar}\vspace*{-0.25cm}
  \]
  Os geradores de translação são, de fato, os operadores momento linear\vspace*{-0.25cm}
  \[
    \begin{array}{ll}
    \ket{\mathbf{p}} = \ket{p_x,p_y,p_z}&\\
    \hat{p}_x\ket{\mathbf{p}} = p_x\ket{\mathbf{p}},\quad \hat{p}_y\ket{\mathbf{p}} = p_y\ket{\mathbf{p}},\quad \hat{p}_z\ket{\mathbf{p}} = p_z\ket{\mathbf{p}}&
    \end{array}\vspace*{-0.25cm}
  \]
  A representação do momento linear na base das posições\vspace*{-0.25cm}
  \[
    \braket{\mathbf{r}|\hat{\mathbf{p}}|\psi} = \dfrac{\hbar}{i}\boldsymbol{\nabla}\braket{\mathbf{r}|\psi}
  \]  
  \end{columns}

\end{frame}



%%%%%%%%%%%%%%%%  SLIDE 9 %%%%%%%%%%%%%%%%%%%%%%%%%%%

\begin{frame}
  \frametitle{Outra notação}
  \fontsize{7pt}{11pt}\selectfont
  
  \begin{columns}[T]
  \column{0.5\linewidth}
    Escolhendo $\ket{\psi}=\ket{\mathbf{p}}$\vspace*{-0.25cm}  
    \[
      \braket{\mathbf{r}|\hat{\mathbf{p}}} = \dfrac{1}{(2\pi \hbar)^{3/2}}e^{-i\hat{\mathbf{p}}\cdot \hat{\mathbf{r}}/\hbar}\vspace*{-0.25cm}  
    \]
    Consideremos, agora, um Hamiltoniano que descreve a interação de dois corpos\vspace*{-0.25cm}  
    \[
     \hat{H} = \dfrac{\hat{\mathbf{p}}_1^2}{2m_1} + \dfrac{\hat{\mathbf{p}}_2^2}{2m_2} + V\left(  \left| \hat{\mathbf{r}}_1 - \hat{\mathbf{r}}_2\right|\right)\vspace*{-0.25cm}  
    \]
    uma notação que deve conhecer-se\vspace*{-0.25cm}
    \[
      \ket{\hat{\mathbf{r}}_1,\, \hat{\mathbf{r}}_2} = \ket{\hat{\mathbf{r}}_1} \otimes \ket{\hat{\mathbf{r}}_2}\vspace*{-0.3cm}  
    \]
    passando para o sistema centro de massas\vspace*{-0.3cm}  
    \[
    \begin{align*}
      \hat{\mathbf{r}} &= \hat{\mathbf{r}}_1 - \hat{\mathbf{r}}_2\\
      \mu &= \dfrac{m_1m_2}{m_1+m_2}\\
      M &= m_1+m_2\\
      \hat{\mathbf{R}} &= \dfrac{m_1\hat{\mathbf{r}}_1 + m_2\hat{\mathbf{r}}_2}{m_1+m_2}\\
      \hat{\mathbf{P}} &= \hat{\mathbf{p}}_1 + \hat{\mathbf{p}}_2\\
      \hat{\mathbf{p}} &= \dfrac{m_2\hat{\mathbf{p}}_1 + m_1\hat{\mathbf{p}}_2}{m_1+m_2}
    \end{align*}\vspace*{-0.25cm}   
    \]    
  
  \column{0.5\linewidth}\vspace*{-0.8cm}
  
    assim
    \[
     \hat{H}=\dfrac{\hat{\mathbf{P}}^2}{2M} + \dfrac{\hat{\mathbf{p}}^2}{2\mu} + V\left(  \left| \hat{\mathbf{r}} \right|\right)\vspace*{-0.25cm}  
    \]
    disso  \vspace*{-0.25cm}  
    \[
      \hat{H}_{cm} =\dfrac{\hat{\mathbf{P}}^2}{2M}, \qquad 
      \hat{H}_{rel} = \dfrac{\hat{\mathbf{p}}^2}{2\mu} + V\left(  \left| \hat{\mathbf{r}}\right|\right)\vspace*{-0.25cm}
    \]
    como esse operadores comutam um com outro, eles possuem auto estado em comum\vspace*{-0.25cm}  
    \[
      \begin{align*}
        \hat{H}\ket{E_{cm},\,E_{rel}} &=\left( \hat{H}_{cm} + \hat{H}_{rel}\right)\ket{E_{cm},\,E_{rel}}\\
        &=\left( E_{cm} + E_{rel}\right)\ket{E_{cm},\,E_{rel}}
      \end{align*}\vspace*{-0.25cm}   
    \]
    Dada a simetria do problema que esperamos tratar, $V(r)$ é invariante frente uma rotação. Definimos o operador de rotação infinitesimal\vspace*{-0.25cm}
    \[
      \hat{R}\left( d\phi\, \mathbf{k}\right) = 1-\dfrac{i}{\hbar}\hat{L}_z\,d\phi\vspace*{-0.25cm}
    \]
    \[
      \begin{align*}
        \left[L_z,\, \hat{p}_x\right]=i\hbar \hat{p}_y,\quad &
        \left[L_z,\, \hat{p}_y\right]=-i\hbar \hat{p}_x,\\
        \left[L_z,\, \hat{p}_z\right]=0,\quad &
        \left[L_z,\, \hat{\mathbf{p}}^2\right]=0\\
        \left[L_z,\, \hat{x}\right]=i\hbar \hat{y},\quad &
        \left[L_z,\, \hat{y}\right]=-i\hbar \hat{x},\\
        \left[L_z,\, \hat{z}\right]=0,\quad &
        \left[L_z,\, \hat{\mathbf{r}}^2\right]=0\\
      \end{align*}\vspace*{-0.25cm}
  \]


  \end{columns}

\end{frame}

%%%%%%%%%%%%%%%%  SLIDE 10 %%%%%%%%%%%%%%%%%%%%%%%%%%%

\begin{frame}
  \frametitle{Outra notação}
  \fontsize{7pt}{11pt}\selectfont
  
  \begin{columns}[T]
  \column{0.5\linewidth}
  
  Aproveitando a simetria, resulta interessante mudar de coordenadas\vspace*{-0.5cm}
  \[
   \ket{x,y,z} = \ket{r,\theta,\phi}\vspace*{-0.5cm}
  \]
  onde é obvio\vspace*{-0.25cm}
  \[
    \hat{R}\left( d\phi\, \mathbf{k}\right)\ket{r,\theta,\phi} = \ket{r,\theta,\phi + d\phi}\vspace*{-0.5cm}
  \]
  assim\vspace*{-0.25cm}
  \[
    \begin{align*}
      \hat{R}\left( d\phi\, \mathbf{k}\right) \, V\left(  \left| \hat{\mathbf{r}}\right|\right) \, \ket{r,\theta,\phi} &= \hat{R}\left( d\phi\, \mathbf{k}\right)V(r)\ket{r,\theta,\phi}\\
      &=V(r)\ket{r,\theta,\phi + d\phi}\\
      &=V\left(  \left| \hat{\mathbf{r}}\right|\right) \, \hat{R}\left( d\phi\, \mathbf{k}\right)\ket{r,\theta,\phi}
    \end{align*}\vspace*{-0.25cm}
  \]
  de forma que $V\left(  \left| \hat{\mathbf{r}}\right|\right)$ e $\hat{R}$ comutam devido a que um só afeta a parte radial e outro a parte angular
  \[
   \left[\hat{H},\, \hat{L}_x\right]= \left[\hat{H},\, \hat{L}_y\right] = \left[\hat{H},\, \hat{L}_z\right] = 0\vspace*{-0.25cm}
  \]
  \[
   \left[\hat{L}_i,\, \hat{L}_j\right] = i\hbar \sum_{k=1}^3\varepsilon_{ijk}\hat{L}_k, \quad  \hat{L}_i = \sum_{j=1}^3 \sum_{k=1}^3\varepsilon_{ijk}\hat{x}_j \hat{p}_k \vspace*{-0.25cm}
  \]  
  Por tanto, podemos pensar em um conjunto de auto estado simultâneos de $\hat{H}$ e $\hat{\mathbf{L}}^2$ e uma das componentes de $L$ que denomina-se $\hat{L}_z$, chamaremos esses auto estados de $\ket{E,\,l,\, m}$
  
  \column{0.5\linewidth}\vspace*{-0.8cm}
  
  pode se verificar,\vspace*{-0.35cm}
  \[
    \begin{align*}
      \hat{H}\ket{E,\,l,\, m} &= E\ket{E,\,l,\, m}\\
      \hat{\mathbf{L}}^2\ket{E,\,l,\, m} &= l\left(l+1\right)\hbar^2\ket{E,\,l,\, m}\\
      \hat{L}_z\ket{E,\,l,\, m} &= m\hbar \ket{E,\,l,\, m}
    \end{align*}\vspace*{-0.25cm}
  \]
  O Hamiltoniano $\hat{H}_{rel}$ pode ser reescrito em termos de $\hat{\mathbf{L}}^2$, usando\vspace*{-0.35cm}
  \[
    \hat{\mathbf{L}}^2 = \hat{\mathbf{r}}^2\hat{\mathbf{p}}^2 - \left( \hat{\mathbf{r}} \cdot \hat{\mathbf{p}} \right)^2 + i\hbar \hat{\mathbf{r}} \cdot \hat{\mathbf{p}}\vspace*{-0.25cm}
  \]
  Vamos aplicar o operador momento à função $\ket{\psi}$ e expressar na base das posições,\vspace*{-0.35cm}
  \[
    \begin{align*}
      \braket{\hat{\mathbf{r}}|\hat{\mathbf{r}}^2\hat{\mathbf{p}}^2|\psi} &= \braket{\hat{\mathbf{r}}|\left( \hat{\mathbf{r}} \cdot \hat{\mathbf{p}} \right)^2|\psi} -i\hbar \braket{\hat{\mathbf{r}}|\hat{\mathbf{r}} \cdot \hat{\mathbf{p}}|\psi}\\
      &\quad+ \braket{\hat{\mathbf{r}}|\hat{\mathbf{L}}^2|\psi}
    \end{align*}\vspace*{-0.25cm}
  \]
  analisando cada termo\vspace*{-0.25cm}
  \[
    \begin{align*}
      \braket{\hat{\mathbf{r}}|\hat{\mathbf{r}}^2\hat{\mathbf{p}}^2|\psi} &= r^2\braket{\hat{\mathbf{r}}|\hat{\mathbf{p}}^2|\psi}\\
      \braket{\hat{\mathbf{r}}|\hat{\mathbf{r}} \cdot \hat{\mathbf{p}}|\psi} &=\mathbf{r} \cdot \dfrac{\hbar}{i}\boldsymbol{\nabla}\braket{\mathbf{r}|\psi}\\
      &=\dfrac{\hbar}{i} \,r\, \dfrac{\partial\,}{\partial r}\braket{\mathbf{r}|\psi}\\
      \braket{\hat{\mathbf{r}}|\left( \hat{\mathbf{r}} \cdot \hat{\mathbf{p}} \right)^2|\psi} &= -\hbar^2\,r\, \dfrac{\partial\,}{\partial r}\left(r\, \dfrac{\partial\,}{\partial r}\right)\braket{\mathbf{r}|\psi}
    \end{align*}\vspace*{-0.25cm}
  \]

  \end{columns}

\end{frame}


%%%%%%%%%%%%%%%%  SLIDE 10 %%%%%%%%%%%%%%%%%%%%%%%%%%%

\begin{frame}
  \frametitle{Outra notação}
  \fontsize{7pt}{11pt}\selectfont
  
  \begin{columns}[T]
  \column{0.5\linewidth}
  
  Como
  \[
    \hat{H}_{rel} = \dfrac{\hat{\mathbf{p}}^2}{2\mu} + V\left(  \left| \hat{\mathbf{r}}\right|\right)\vspace*{-0.15cm}
  \]
  onde, o primeiro termo pode ser modificado
  \[
    \begin{align*}
      \Braket{\mathbf{r}|\dfrac{\hat{\mathbf{p}}^2}{2\mu}|\psi} &= -\dfrac{\hbar^2}{2\mu} \left( \dfrac{\partial^2\, }{\partial r^2} + \dfrac{2}{r}\dfrac{\partial \, }{\partial r} \right)\braket{\mathbf{r}|\psi}\\
      &\;\quad +\dfrac{\braket{\hat{\mathbf{r}}|\hat{\mathbf{L}}^2|\psi}}{2\mu r^2}
    \end{align*}\vspace*{-0.25cm}
  \]
  assim, o operador Hamiltoniano\vspace*{-0.25cm}
  \[
    \begin{align*}
      &\braket{\mathbf{r}|\hat{H}|\psi} \\
      &= -\dfrac{\hbar^2}{2\mu} \left( \dfrac{\partial^2\, }{\partial r^2} + \dfrac{2}{r}\dfrac{\partial \, }{\partial r} \right)\braket{\mathbf{r}|\psi} +\dfrac{\braket{\hat{\mathbf{r}}|\hat{\mathbf{L}}^2|\psi}}{2\mu r^2}\\
      &\;\quad + \braket{\mathbf{r}| V\left(  \left| \hat{\mathbf{r}}\right|\right) |\psi}
    \end{align*}\vspace*{-0.25cm}
  \]
  Esta expressão têm duas partes, uma parte é a energia cinética de rotação $\hat{\mathbf{L}}^2/2I$, onde $I=\mu r^2$ é o momento de inercia e o outro termo é a parte radial:
  \[
   \braket{\mathbf{r}|\hat{p}_r^2|\psi} = -\dfrac{\hbar^2}{2\mu} \left( \dfrac{\partial^2\, }{\partial r^2} + \dfrac{2}{r}\dfrac{\partial \, }{\partial r} \right)\vspace*{-0.25cm}
  \]
  
  \column{0.5\linewidth}\vspace*{-0.8cm}
  
  de onde
  \[
   \hat{p}_r \to \dfrac{\hbar}{i} \left( \dfrac{\partial \, }{\partial r}  +
   \dfrac{1}{r}\right)\vspace*{-0.25cm}
  \]
  usando $\ket{\psi}=\ket{E,\,l,\, m}$
  \[
    \begin{align*}
      \left[  -\dfrac{\hbar^2}{2\mu} \left( \dfrac{\partial^2\, }{\partial r^2} + \dfrac{2}{r}\dfrac{\partial \, }{\partial r} \right) \right. \qquad \qquad\qquad \qquad&\\
    \left. + \dfrac{l(l+1)\hbar}{2\mu r^2} + V(r)\right]\braket{\mathbf{r}|E,\,l,\, m} &\\
   = E\braket{\mathbf{r}|E,\,l,\, m} &
    \end{align*}\vspace*{-0.25cm}
  \]


  \end{columns}

\end{frame}
  

\end{document}
